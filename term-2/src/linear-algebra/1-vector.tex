\section{Векторные пространства}
\subsection{Определение}

\begin{definition} 
    $K$ — поле; \textbf{\textit{векторное пространство над}} $K$ — это 
тройка $(V,+,\cdot)$, где $V$ — множество, $+:V\times V\rightarrow V; 
\cdot :K\times V\rightarrow V$. Элементы $V$ — векторы, элементы $K$ — 
скаляры. \\
    При этом выполняются \textit{аксиомы}:
    \begin{enumerate}
        \item[1-4)] $(V,+)$ — абелева группа.
        \item[5)]  $\forall k_1, k_2\in K,\ \forall v\in V:\ (k_1\cdot 
k_2)\cdot v = k_1\cdot (k_2\cdot v)$
        \item[6)]  $\forall k_1,k_2\in K,\ \forall v\in V:\ 
k\cdot(v_1+v_2) = k\cdot v_1+ k\cdot v_2$
        \item[7)] $\forall k\in K,\ \forall v_1,v_2\in V:\ (k_1+k_2)\cdot 
v = k_1\cdot v+ k_2\cdot v$
        \item[8)] $\forall v\in V,\ 1\in K:\ 1\cdot v=v$
    \end{enumerate}
\end{definition}

\begin{remark}Очевидные свойства:
    \begin{enumerate}
        \item $0\cdot v= \overline{0}$
        \item $(-1)\cdot v=-v$
        \item $a+b=b+a$ следует из 7-ми аксиом.
    \end{enumerate}
\end{remark}

\begin{proof} 
    $2)\ 1\cdot v+(-1)\cdot v=(1+(-1))\cdot v=0\cdot 
v=\overline{0}\Rightarrow (-1)\cdot v$ — противоположный к $v$.
    
    1, 3 TODO proof
\end{proof}
\begin{example}
\begin{enumerate}
    \item[]
    \item векторы на плоскости — класс эквивалентности направленных 
отрезков $\rightarrow$ в.п.
    \item арифметическое векторное пространство  
$K^n=\bigg\{\begin{pmatrix}
    k_1\\ k_2 \\...\\ k_n
    \end{pmatrix}\ \bigg |\ k_i\in K\bigg\}$

    Для него выполняются операции:

    $\begin{pmatrix}
        a_1\\...\\ a_n 
    \end{pmatrix}+\begin{pmatrix}
        b_1\\...\\ b_n
    \end{pmatrix}=\begin{pmatrix}
    a_1+b_1\\...\\ a_n+b_n
    \end{pmatrix}$
    
    $k\begin{pmatrix}
        a_1\\...\\ a_n
    \end{pmatrix}=\begin{pmatrix}
        ka_1\\...\\ ka_n
    \end{pmatrix}$
    
    $^nK=\{(k_1,...,k_n)\|\ k_i\in K\}$
    
    $K^n$ и $^nK$ изоморфны: $\exists$ биекция $f:K^n\rightarrow\  ^nK$, 
является гомоморфизмом.
    
\end{enumerate}
\end{example}

\begin{definition}$u$ и $v$ — векторные пространства над K; 
$f:u\rightarrow v$ называется \textbf{\textit{гомоморфизмом (линейным 
отображением)}}, если:
\begin{enumerate}
    \item $f(a+b)=f(a)+f(b)$  
    \item $f(k\cdot b)=k\cdot f(b)$
\end{enumerate}
\end{definition}

\begin{example}
    \begin{enumerate}
        \item[]
        \item $K[x]$ — векторное пространство над $K$
        \item $K[x^n]=\{f\ |\ deg\ f\leq n\}$ — в.п. над $K$
        \item Пусть $R$ — кольцо, $K_{\text{поле}}\subset R$ — подкольцо; 
если $r_1+r_2$ и $kr$ определено $\forall r\in R$, то $R$ — векторное 
пространство над $K$
        \item $\mathbb{C}$ — векторное пространство над $\R$
        \item $\mathbb{C}$ — векторное пространство над $\mathbb{Q}$
        \item $M$ — множествово; $V=\{f:M\rightarrow K\}$ — векторное 
пространство над $K$ (так как можно определить 
$(f_1+f_2)(m):=f_1(m)+f_2(m)$ и $(k\cdot f)(m):=k\cdot f(m))$
    \end{enumerate}
\end{example}

\begin{example}
    \begin{enumerate}
        \item[]
        \item[] $M=k=\R$; 
        \item $\text{Func}(\R,\R)$
        \item $\text{Func}_c(\R,\R)$ — непрерывные функции, в.п. над $\R$
    \end{enumerate}
\end{example}

\subsection{Фибоначчиевы последовательности}

\begin{definition}
    \textbf{\textit{Последовательность фибоначчиева}}, если для нее 
выполняется, что $\forall n\ a_n =a_{n-1}+a_{n-2}$. При этом $a_n+b_n$ и 
$ka_n$ тоже фибоначчиевы $\Rightarrow$  фибоначчиевы последовательности 
векторного пространства.
\end{definition}

\begin{example}
    \begin{enumerate}
        \item[]
        \item $M$ — множество; $V=2^M; K=\Z/_2\Z$ — в.п. \\
        $X+Y:=(X\cup Y)\setminus (X\cap Y)$ \\    $\varnothing$ — 
нейтральный, $0\cdot x:=0,\ 1\cdot x:=x$ \\
        Замкнуто: $(1+1)\cdot x=1\cdot x+ 1\cdot x=x+x=\varnothing$. 
Важно, что $1+1=0$ в $K$, т.е. $char\ K=2$
    \end{enumerate}
\end{example}

\begin{example}
    \begin{enumerate}
        \item[]
        \item[] \textit{Как ввести координаты:}
        \item $K[x]_n=\{a_0+a_1x+...+a_nx^n\}=(a_0,...,a_n)$ 
        
        $K[x]_n\cong K^{n+1}$
        \item $\mathbb{C}$ над $\R:z=a+bi\leadsto (a,b)$
        
        $\mathbb{C}\cong\R^2$
        \item Пример плохих координат: \\ $z\leadsto (r,\varphi)$ — не 
согласуется с операциями
        \item $\text{Func}(\N,\R)\cong \R^n\leadsto (a_0,...,)$
        \item  фиб. пос-ти $(a_1,a_2,...,)\leadsto(a_1,a_2)$
        
        фиб. пос-ти $\cong \R^2$
        \item $N\subset M=\{a_1,...,a_n\}$ \\    $N\leadsto 
(\varepsilon_1,..., \varepsilon_n),\ \varepsilon_i=\begin{cases}1,a_i\in N 
\\ 0,a_i\notin N\end{cases}$
    \end{enumerate}
\end{example}
