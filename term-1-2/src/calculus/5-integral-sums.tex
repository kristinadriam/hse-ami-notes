\section{Интегральные суммы}
\begin{definition}
    $f:E\rightarrow \R;\ f$ равномерно непрерывна, если $\forall \varepsilon\ \exists \delta(\varepsilon) >0:\ \forall x,y\in E:|x-y|<\delta\Rightarrow |f(x)-f(y)|<\varepsilon$ 
\end{definition}

\begin{remark}
    $f:E\rightarrow \R;\ f$  непрерывна во всех точках, если $\forall x\in E\ \forall \varepsilon>0\ \exists \delta(x,\varepsilon) >0:\forall y\in E:|x-y|<\delta\Rightarrow |f(x)-f(y)|<\varepsilon$ 
\end{remark}

\begin{statement}
    $f:E\rightarrow \R$ равномерно непрерывна $\Rightarrow f$ непрерывна на $E$.
\end{statement}

\begin{example}
    \begin{enumerate}
        \item[]
        \item[0.] $f(x)=x$
        \item $f(x) = \sin x$ равномерно непрерывна на $\R:\ |\sin x-\sin y|\leq |x-y|,\ \delta=\varepsilon$ подходит
        \item $f:E\rightarrow \R$ липшицева с константой $L$, если $\forall x,y\in E\ |f(x)-f(y)|\leq L\cdot|x-y|$ 
        
        $f$ равномерно непрерывна на $E:\ \delta=\frac{\varepsilon}{L}$ подходит
        \item $f(x)=x^2$ \underline{не} равномерно непрерывна на $\R$

        Возьмем $\varepsilon=1$ и проверим, что никакое $\delta>0$ не подходит:

        $y=x+\frac{\delta}{2},\ |x-y|<\delta;\ f(x)-f(y)=y^2-x^2=(x+\frac{\delta}{2})^2-x^2=x\delta+\frac{\delta^2}{4}>x\delta >1$ при $x>\frac{1}{\delta}$
        \item $f(x)=\frac{1}{x}$ на $(0,+\infty)$ \underline{не} равномерно непрерывна

        Возьмем $\varepsilon=1$ и проверим, что никакое $\delta>0$ не подходит:

        $y=\frac{\delta}{2}$ и $x=\delta,\ |x-y|<\delta;\ f(y)-f(x)=\frac{2}{\delta}-\frac{1}{\delta}=\frac{1}{\delta}>1$ при $\delta<1$ не подходит.

        Если $\delta\geq 1$, то $y=\frac{1}{2},\ x=1,$ то $f(y)-f(x)=1$.
    \end{enumerate}
\end{example}

\begin{theorem} \textbf{Теорема Кантора}

$f:[a,b]\rightarrow \R$ непрерывна во всех точках; тогда $f$ равномерно непрерывна на $[a,b]$.
\end{theorem}

\begin{proof}
    Зафиксируем $\varepsilon>0$ и предположим, что никакое $\delta>0$ не подходит. В частности, $\delta =1$ не подходит $\Rightarrow \exists x_1,y_1\in [a,b]: |x_1-y_1|<\delta$ и $|f(x_1)-f(y_1)|\geq \varepsilon$

    $\delta =\frac{1}{2}$ не подходит $\Rightarrow \exists x_2,y_2\in [a,b]: |x_2-y_2|<\delta$ и $|f(x_2)-f(y_2)|\geq \varepsilon$

    ...

    $\delta =\frac{1}{n}$ не подходит $\Rightarrow \exists x_n,y_n\in [a,b]: |x_n-y_n|<\delta$ и $|f(x_n)-f(y_n)|\geq \varepsilon$

    $x_n\ -$ ограниченная последовательность $\overset{\text{Б.-В.}}{\Rightarrow} $ можно выбрать сходящуюся подпоследовательность $x_{n_k}\rightarrow c$

    $a\leq x_{n_k}\leq b\Rightarrow c\in [a,b];\ f$ непрерывна в $c\Rightarrow f(x_{n_k})\rightarrow f(c)$

    $c\leftarrow x_{n_k}-\frac{1}{n_k}< y_{n_k}<x_{n_k}+\frac{1}{n_k}\rightarrow c\Rightarrow y_{n_k}\rightarrow c\Rightarrow f(y_{n_k})\rightarrow f(c)$

    Следовательно $f(x_{n_k})-f(y_{n_k})\rightarrow f(c)-f(c)=0$, но $|f(x_{n_k})-f(y_{n_k})|\geq \varepsilon$, получили противоречие.
\end{proof}

\begin{definition}
    $f:E\rightarrow \R;\ \omega_f(\delta):=\sup \{|f(x)-f(y)| : x,y\in E $ и $ |x-y|<\delta \}$ для $\delta\geq 0$.

    $\omega_f(\delta)\ -$ модуль непрерывности.
\end{definition}

\begin{statement} \textbf{Свойства:}
    \begin{enumerate}
        \item $\omega_f(0)=0$.
        \item $\omega_f\geq 0$.
        \item $\omega_f$ нестрого возрастает.
        \item $|f(x)-f(y)|\leq \omega_f(|x-y|)$.
        \item Если $f$ липшицева с константой $L$, то $\omega_f(\delta)\leq L\delta$.

        $(|f(x)-f(y)|\leq L|x-y|\leq L\delta$, если $|x-y|\leq \delta$; то есть все числа (и $\sup$ в том числе) не превышают $L\delta)$
        \item $f$ равномерно непрерывна на $E\Leftrightarrow \omega_f$ непрерывна в нуле (то есть $\lim\limits_{\delta\rightarrow0_+}\omega_f(\delta)=0$).
        \begin{proof}
            $\Rightarrow:\ f$ равномерно непрерывна; возьмем $\varepsilon>0\Rightarrow \exists \delta >0\forall x,y\in E$ и $|x-y|<\delta\Rightarrow |f(x)-f(y)|<\varepsilon\Rightarrow$ если $|x-y|\leq\frac{\delta}{2}$, то $|f(x)-f(y)|<\varepsilon\Rightarrow\omega_f(\frac{\delta}{2})\leq \varepsilon\Rightarrow w_f(\alpha)\leq\varepsilon \forall \alpha <\frac{}{\delta}{2}\Rightarrow \lim\limits_{\delta\rightarrow0_+}\omega_f(\delta)=0$

            $\Leftarrow:$ возьмем $\varepsilon>0$ и такую $\delta>0$, что $\omega_f(\delta)<\frac{\varepsilon}{2}\Rightarrow $ если $|x-y|<\delta$, то $|f(x)-f(y)|\leq \frac{\varepsilon}{2}<\varepsilon$
        \end{proof}
        \item $f\in C[a,b]\Leftrightarrow \lim\limits_{\delta\rightarrow0_+}\omega_f(\delta)=0$

        $(\Rightarrow:$ th Кантора; $\Leftarrow:$ 6 свойство$)$
    \end{enumerate}
\end{statement}

\begin{definition}
    $[a,b]:\ a=x_0<x_1<...<x_n=b;\ \tau=\{x_0,...,x_n\}$

    $\tau\ -$ дробление (разбиение, пунктир) отрезка $[a,b]$. 
\end{definition}

\begin{definition}
    Ранг дробления $|\tau|:=\max\{x_1-x_0,...,x_n-x_{n-1}\}$ (самый длинный отрезок дробления).
\end{definition}

\begin{definition}
    Оснащение дробления $\xi=\{\xi_1,...,\xi_n\},\ \xi_k\in[x_{k-1},x_k]\ \forall k$.
\end{definition}

\begin{definition}
    Интегральная сумма (сумма Римана): $S(f,\tau,\xi)=\sum\limits_{k=1}^n f(\xi_k)(x_k-x_{k-1})$
\end{definition}

\begin{theorem} \textbf{Теорема об интегральных суммах}

    $f\in C[a,b];$ тогда $\bigg|\int\limits_{a}^b f-S(f,\tau,\xi)\bigg|\leq (b-a)\cdot \omega_f(|\tau|)$
\end{theorem}

\begin{proof}
    $\Delta:=\int\limits_{a}^b f-S(f,\tau,\xi)
    =\int\limits_{a}^b f-\sum\limits_{k=1}^n f(\xi_k)(x_k-x_{k-1}) 
    =\sum\limits_{k=1}^n \int\limits_{x_{k-1}}^{x_k} f-\sum\limits_{k=1}^n f(\xi_k)(x_k-x_{k-1})
    =\sum\limits_{k=1}^n \bigg(\int\limits_{x_{k-1}}^{x_k} f- f(\xi_k)(x_k-x_{k-1})\bigg)
    =\sum\limits_{k=1}^n \int\limits_{x_{k-1}}^{x_k} \bigg(f(t)-f(\xi_k)\bigg)\,dt$

    $|\Delta|\leq \sum\limits_{k=1}^n \int\limits_{x_{k-1}}^{x_k} \underbrace{\bigg|f(t)-f(\xi_k)\bigg|}_{\leq\omega_f(x_k-x_{k-1})\leq \omega_f(|\tau|)}\,dt\leq \sum\limits_{k=1}^n \int_{x_{k-1}}^{x_k} \omega_f(|\tau|)=\omega_f(|\tau|)\sum\limits_{k=1}^n(x_k-x_{k-1})=\omega_f(|\tau|)(b-a)$
\end{proof}

\begin{corollary}
    \begin{enumerate}
        \item[]
        \item $\forall\varepsilon>0\ \exists \delta>0\ \forall$ дробления ранга $<\delta$ и $\forall$ его оснащения: $\bigg|\int\limits_{a}^b f-S(f,\tau,\xi)\bigg|<\varepsilon$.
        \item $\tau_n\ -$ последовательность дроблений, т.ч. если $|\tau_n|\rightarrow 0$, то $S(f,\tau,\xi)\rightarrow \int\limits_{a}^b f$.
    \end{enumerate}
\end{corollary}

\begin{example}
    $S_p(n):=1^p+2^p+...+n^p,\ p\geq 0$

    Хотим посчитать $\lim\frac{S_p(n)}{n^{p+1}}=\frac{1}{p+1}$.

    Возьмем непрерывную $f(x)=x^p$ и воспользуемся теоремой для нее:

    $\frac{S_p(n)}{n^{p+1}}=\frac{1}{n}\sum\limits_{k=1}^n f(\frac{k}{n})=S(f, \tau, \xi)\rightarrow \int\limits_0^1 f(x)\,dx=\int\limits_0^1 x^p\,dx=\left.\frac{x^{p+1}}{p+1}\right|_0^1=\frac{1}{p+1}$

    $x_k=\frac{k}{n},\ [a,b]=[0,1],\ x_k-x_{k+1}=\frac{1}{n}.\ \xi_k=x_k$
\end{example}

\begin{definition}
    $f:[a,b]\rightarrow \R,\ f\ -$ интегрируема по Риману и $I\ -$ ее интеграл, если $\forall \varepsilon >0\ \exists\delta>0:\forall$ дробления ранга $<\delta$ и $\forall$ оснащения $\bigg|S(f,\tau, \xi) - I\bigg|<\varepsilon$.
\end{definition}

\begin{remark}
    Любая непрерывная функция $-$ такая.
\end{remark}

\begin{remark}
    Берем дробления на равные отрезки $x_k-x_{k-1}=\frac{b-a}{n}:\ x_k=a+k\cdot \frac{b-a}{n}$ и $\xi_k=x_k$.

    $S(f,\tau, \xi)=\sum\limits_{k=1}^nf(x_k)\cdot\frac{b-a}{n}=\frac{b-a}{n}\cdot \sum\limits_{k=1}^nf(x_k)\rightarrow \int\limits_a^b$

    Теперь рассмотрим $\xi'_k=x_{k-1}:S(f,\tau, \xi')=\sum\limits_{k=1}^nf(x_{k-1})\cdot\frac{b-a}{n}=\frac{b-a}{n}\cdot \sum\limits_{k=0}^{n-1}f(x_k)\rightarrow \int\limits_a^b$

    Сумма площадей трапеций: $\sum\limits_{k=1}^n\frac{f(x_{k-1}+x_k)}{2}(x_k-x_{k-1})=\frac{b-a}{n}\cdot \sum\limits_{k=1}^n\frac{f(x_{k-1}+x_k)}{2}
    =\frac{b-a}{n}\cdot\bigg(\frac{f(x_0)}{2}+\frac{x_n}{2}+\sum\limits_{k=1}^{n-1}f(x_k)\bigg)$
\end{remark}

\begin{lemma}
    $f\in C^2[\alpha, \beta];$ тогда $\int\limits_\alpha^\beta f-\frac{f(\alpha)+f(\beta)}{2}(\beta-\alpha)=-\frac{1}{2}\int\limits_\alpha^\beta f''(t)(t-\alpha)(\beta-t)dt$
\end{lemma}

\begin{proof}
    $\gamma := \frac{\alpha + \beta}{2}$; $\int\limits_\alpha^\beta f'(t)(t-\gamma)dt=\underbrace{\left. f(t)(t-\gamma)\right|^\beta_\alpha}_{f(\beta)\cdot\frac{\beta-\alpha}{2}-f(\alpha)\cdot(-\frac{\beta-\alpha}{2})=\frac{f(\beta)+f(\alpha)}{2}(\beta-\alpha)} -\int\limits_\alpha^\beta f'(t)(t-\gamma)dt$

    Рассмотрим $\bigg((t-\alpha)(\beta-t)\bigg)'=\bigg(-t^2+(\beta+\alpha)t-\alpha\beta\bigg)'=-2t+(\beta+\alpha)=-2(t-\gamma)$

    $\Delta=-\int\limits_\alpha^\beta f'(t)(t-\gamma)dt
    =\frac{1}{2}\int\limits_\alpha^\beta f'(t)\bigg((t-\alpha)(\beta-t)\bigg)'dt
    =\underbrace{\left.\frac{1}{2}f'(t)(t-\alpha)(\beta-t)\right|^\beta_\alpha}_{0}-\frac{1}{2}\int\limits_\alpha^\beta f''(t)(t-\alpha)(\beta-t)dt$
\end{proof}

\begin{theorem}
    \textbf{Оценка погрешности в формуле трапеций}

    $f\in C^2[a,b]$; тогда $\bigg|\int\limits_\alpha^\beta f-\sum\limits_{k=1}^n\frac{f(x_{k-1})+f(x_k)}{2}(x_k-x_{k-1})\bigg|\leq \frac{|\tau|^2}{8}\cdot \int\limits_\alpha^\beta|f''|$
\end{theorem}

\begin{proof}
    $\Delta =\int\limits_\alpha^\beta -\sum\limits_{k=1}^n
    =\sum\limits_{k=1}^n\bigg(\int\limits_{x_{k-1}}^{x_k} f - \frac{f(x_k)+f(x_{k-1})}{2}\cdot(x_k-x_{k-1})\bigg)
    =-\frac{1}{2}\sum\limits_{k=1}^n \int\limits_{x_{k-1}}^{x_k} f'' (t)(t-x_{k-1})(x_k-t)dt$

    $|\Delta|\leq \frac{1}{2}\sum\limits_{k=1}^n \int\limits_{x_{k-1}}^{x_k} |f'' (t)|(t-x_{k-1})(x_k-t)dt\leq \frac{|\tau|^2}{8}\sum\limits_{k=1}^n\int\limits_{x_{k-1}}^{x_k} |f''(t)|dt=\frac{|\tau|^2}{8}\int\limits_{a}^{b} |f''(t)|dt$

    $(t-x_{k-1})(x_k-t)\leq (\frac{x_k-x_{k-1}}{2})^2=\frac{1}{4}(x_k-x_{k-1})^2\leq \frac{|\tau|^2}{4}$
\end{proof}

\begin{theorem}
    \textbf{Формула Эйлера-Маклорена для второй производной}

    $f\in C^2[m,n];$ тогда $\sum\limits_{k=m}^n f(k)=\frac{f(m)+f(n)}{2}+\int\limits_m^n f(t)dt+\frac{1}{2}\int\limits_m^nf''(t)\cdot \{t\}\cdot(1-\{t\})dt$
\end{theorem}

\begin{proof}
    $\int\limits_{k-1}^k f(t)dt-\frac{f(k-1)+f(k)}{2}
    =-\frac{1}{2}\int\limits_{k-1}^k f''(t)\underbrace{(t-(k-1))}_{\{t\}}\underbrace{(k-t)}_{1-\{t\}}dt
    =-\frac{1}{2}\int\limits_{k-1}^k f''(t)\cdot\{t\}\cdot(1-\{t\})dt$

    Суммирование по $k$ от $m+1$ до $n$: $\underbrace{\sum\limits_{k=m+1}^n\int\limits_{k-1}^k f(t)dt}_{\int\limits_m^n f(t)dt}-\underbrace{\sum\limits_{k=m+1}^n\frac{f(k-1)+f(k)}{2}}_{\sum\limits_{k=m}^nf(k)-\frac{f(m)+f(n)}{2}}=$
    
    $=-\frac{1}{2}\underbrace{\sum\limits_{k=m+1}^n\int\limits_{k-1}^k f''(t)\cdot\{t\}\cdot(1-\{t\})dt}_{\int\limits_{m}^n f''(t)\cdot\{t\}\cdot(1-\{t\})dt}$
\end{proof}

\begin{example}
    \begin{enumerate}
        \item[]
        \item $S_p(n):=1^p+2^p+...+n^p,\ f(t)=t^p,\ p>-1,\ m=1,\ n=n,\ f''(t)=p(p-1)\cdot t^{p-2}$
        
        $S_p(n)=\frac{f(1)+f(n)}{2}+\int\limits_1^nf(t)dt+\frac{1}{2}\int\limits_1^nf''(t)\cdot \{t\}\cdot (1-\{t\})dt
        =\frac{1}{2}+\frac{n^p}{2}+\underbrace{\int\limits_1^nt^pdt}_{=\left.\frac{t^{p+1}}{p+1}\right|_1^n=\frac{n^{p+1}}{p+1}-\frac{1}{p+1}}+\frac{1}{2}\int\limits_1^np(p-1)t^{p-2})\{t\}(1-\{t\})dt$

        \begin{enumerate}
            \item[$\circ$] Случай $p\in (-1,1)$: $S_p(n)=\frac{n^{p+1}}{p+1}+\frac{n^p}{2}+\mathcal{O}(1)$

            $0<\int\limits_1^nt^{p-2}\underbrace{\{t\}(1-\{t\})}_{\leq \frac{1}{4}}dt\leq \frac{1}{4}\underbrace{\int\limits_1^nt^{p-2}dt}_{=\left.\frac{t^{p-1}}{p-1}\right|_1^n=\frac{1}{1-p}-\frac{n^{p-1}}{1-p}<\frac{1}{1-p}}$, то есть интеграл оценивается константой
            \item[$\circ$] Случай $p > (-1,1)$: $S_p(n)=\frac{n^{p+1}}{p+1}+\frac{n^p}{2}+o(n^{p-1})$

            $0<\int\limits_1^nt^{p-2}\cdot \{t\}\cdot (1-\{t\})dt\leq \left.\frac{t^{p-1}}{p-1}\right|_1^n=\frac{n^{p-1}-1}{p-1}$
        \end{enumerate}
        
        \item Гармонические числа $H_n:=1+\frac{1}{2}+...+\frac{1}{n}$

        $f(t)=\frac{1}{t},\ m=1,\ n=n,\ f''(t)=\frac{2}{t^3}$

        $H_n=\frac{1+\frac{1}{n}}{2}+\underbrace{\int\limits_1^n\frac{dt}{t}}_{\left.\ln t\right|_1^n=\ln n}+\underbrace{\frac{1}{2}\int\limits_1^n\frac{2\{t\}(1-\{t\})}{t^3}dt}_{:=a_n}$

        $H_n=\ln n + \frac{1}{2}+\frac{1}{2n}+a_n$

        $a_n\ -$ возрастающая последовательность; $a_n\leq \int\limits_1^n\frac{1}{4}\frac{dt}{t^3}=\left.\frac{1}{4}(-\frac{1}{2t^2})\right|_1^n=\frac{1}{4}(\frac{1}{2}-\frac{1}{2n^2})\leq \frac{1}{8}$

        $a_n$ возрастающая и ограниченная $\Rightarrow \exists \lim a_n=a\leq \frac{1}{8}\Rightarrow a_n=a+o(1)$

        $H_n=\ln n + \underbrace{(a+\frac{1}{2})}_{=:\gamma}+o(1)$

        $\gamma\ -$ постоянная Эйлера, $\gamma\approx 0,5772156649...$ 

        \item Формула Стирлинга: $n!\sim n^n\cdot e^{-n}\sqrt{2\pi n}$

        $\ln n!=\sum\limits_{k=1}^n\ln k,\ f(t)=\ln t,\ f''(t) =-\frac{1}{t^2},\ m=1,\ n=n$

        $\ln n!=\underbrace{\frac{\ln 1 + \ln n}{2}}_{=\frac{1}{2}\ln n}+\underbrace{\int\limits_1^n\ln t\ dt}_{=n\cdot \ln n - n*}=\underbrace{\frac{1}{2}\int\limits_1^n \frac{\{t\}(1-\{t\})}{t^2}dt}_{=:b_n}\Rightarrow $

        $*\int\limits_1^n\ln t\ dt=t\cdot \ln t\left.\right|_1^n-\int\limits_1^n t\frac{1}{t} dt=n\cdot \ln n-n$

        $\Rightarrow \ln n!=n\cdot \ln n -n+\frac{1}{2}\ln n -b_n$

        $b_n\ -$ возрастающая последовательность: $b_{n+1}-b_n=\frac{1}{2}\int\limits_n^{n+1}\frac{\{t\}(1-\{t\})}{t^2} dt>0$

        $b_n\ -$ ограниченная последовательность: $b_n\leq\frac{1}{8}\int\limits_1^{n}\frac{dt}{t^2}=\frac{1}{8}(-\frac{1}{t})\left.\right|_1^n=\frac{1}{8}-\frac{1}{8n}<\frac{1}{8}$

        Тогда существует $\lim b_n=b$ и $b_n=b+o(1)$.

        $\ln n!=n\cdot \ln n - b + \frac{1}{2}\ln n-b+o(1)\Rightarrow n!=n^n\cdot e^{-n}\cdot\sqrt{n} \cdot e^{-b}\cdot \underbrace{e^{o(1)}}_{1+o(1)\sim 1}\sim n^n\cdot e^{-n}\cdot\sqrt{n} \cdot C$

        Найдем $C$. Рассмотрим $\frac{4^n}{\sqrt{\pi n}}\sim\binom{2n}{n}=\frac{(2n)!}{(n!)^2}\sim\frac{(2n)^{2n}\cdot e^{-2n}\cdot\sqrt{2n}\cdot C}{(n^n\cdot e^{-n}\sqrt{n}\cdot C)^2}=\frac{2^{2n\cdot \sqrt{2n}\cdot C}}{\sqrt{n}\cdot \sqrt{n} \cdot C^2}=\frac{4^n\cdot\sqrt{2}}{\sqrt{n}\cdot C}$

        Тогда $\frac{4^n}{\sqrt{\pi n}}\sim \frac{4^n\cdot \sqrt{2}}{\sqrt{n}\cdot C}\Leftrightarrow \frac{1}{\sqrt{\pi}}\sim \frac{\sqrt{2}}{C}\Rightarrow C=\sqrt{2\pi}$
        \item Если $\int\limits_a^b f$ сходится, то $\lim\limits_{c\rightarrow b_-}\int_c^b f=0$.
        \begin{proof}
            $\int\limits_a^b f=\underbrace{\int\limits_a^c f}_{\rightarrow \int\limits_a^b f} + \int\limits_c^b f\Rightarrow \int\limits_c^b\rightarrow 0$
        \end{proof}
    \end{enumerate}
\end{example}