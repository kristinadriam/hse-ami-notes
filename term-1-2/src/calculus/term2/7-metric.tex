\part{Метрические пространства}
\section{Метрические и нормированные пространства}

\begin{definition}
    \item $\rho:X\times X\rightarrow [0,+\infty)\ -$ \textbf{метрика} (\textbf{расстояние}), если:
    \begin{enumerate}
        \item $\rho(x,x)=0$ и $\rho(x,y)=0\Rightarrow x=y$.
        \item $\rho(x,y)=\rho(y,x)\ \forall x,y\in X$.
        \item $\rho(x,y)\leq \rho(x,z) + \rho(z,y)\ \forall x,y,z\in X$ (неравенство треугольника).
    \end{enumerate}
\end{definition}

\begin{definition}
    Пара $(X,\rho)\ -$
    ъэто \textbf{метрическое пространство}.
\end{definition}

\begin{example}
    \begin{enumerate}
        \item[]
        \item Дискретная метрика: $\rho(x,x)=0,\ \rho(x,y)=1$, если $x\neq y$
        \item $X=\R,\ \rho(x,y)=|x-y|$
        \item $X=\R^2$ расстояние на плоскости
        \item Манхэттеновская метрика: $X=\R^2,\ \rho((x_1, y_1), (x_2, y_2))=|x_1-x_2|+|y_1+y_2|$
        \item $\R^d=(x_1,...,x_d)$

        $\rho(x,y)=\sqrt{(x_1-y_1)^2+...+(x_d-y_d)^2}^*$ (на самом деле, даже $p$-ая степень и корень $p$-ой степени подойдет)

        $^*$Неравенство треугольника в этом случае $-$ это неравенство Минковского.
        \item $X=C[a,b],\ \rho(f,g)=\int\limits_a^b |f-g|$.
        \item Французская железнодорожная метрика: $\rho(A,B)=AB$, если $A$ и $B$ на одной прямой и $\rho(AB)=AP+PB$ иначе.

        (картиночка)
    \end{enumerate}
\end{example}

\begin{definition}
    $(X,\rho)\ -$ метрическое пространство, $a\in X,\ r>0$.

    \textbf{Открытый шар} $B_r(a):=\{x\in X\mid \rho(x,a)<r\}$.

    \textbf{Замкнутый шар} $\overline{B}_r(a):=\{x\in X\mid \rho(x,a)\leq r\}$.

    $a\ -$ центр шара, $r\ -$ радиус шара.
\end{definition}

\begin{statement}
    \textbf{Свойства:}
    \begin{enumerate}
        \item $B_{r_1}(a)\cap B_{r_2}(a)=B_{\min\{r_1, r_2\}}(a)$.
        \item Если $a\neq b$, то $\exists r>0:\overline{B}_r(a)\cap \overline{B}_r(b)=\varnothing$.
    \end{enumerate}
\end{statement}

\begin{proof}
    $r:=\frac{\rho(a,b)}{3}>0.$ Предположим, что $x\in \overline{B}_r(a)\cap \overline{B}_r(b)\Rightarrow \rho(x,a)\leq r\And \rho(x,b)\leq r\Rightarrow \rho(a,b)\leq \rho(a,x)+\rho(x,b)\leq r+r=\frac{2}{3}\rho(a,b)$, противоречие.
\end{proof}

\begin{definition}
    $A\subset X;\ A\ -$ \textbf{открытое множество}, если $\forall a\in A$ найдется $B_r(a)\subset A$.
\end{definition}

\begin{theorem}
    \textbf{О свойствах откртых множеств}

    \begin{enumerate}
        \item $\varnothing$ и $X\ -$ открытые множества.

        \item Объединение любого числа открытых множеств $-$ открытое множество.

        \item Пересечение конечного числа открытых множеств $-$ открытое множество.

        \item $B_r(a)\ -$ открытое множество.
    \end{enumerate}
\end{theorem}

\begin{remark}
    В третьем конечность существенна: $\cap\limits_{n=1}^\infty (-1,\frac{1}{n})=(-1,0]$.
\end{remark}

\begin{proof}
    \begin{enumerate}
        \item[]
        \item[2.] $A_\alpha\ -$ открытые множества, $\alpha\in I$. Проверим, что $U:=\cup\limits_{n=1}A\ -$ открытое.

        Возьмем $a\in U\Rightarrow$ найдется $\alpha_0:a\in A_{\alpha_0}\ -$ открытое $\Rightarrow $ найдется $r>0:B_r(a)\subset A_\alpha \subset U$.

        \item[3.] $A_1,...,A_n\ -$ открытые множества. Проверим, что $U:=\cap\limits_{k=1}^n A_k\ -$ открытое.

        Возьмем $a\in U\Rightarrow a\in A_k\ k=1,..,n\ -$ открытое $\Rightarrow $ найдется такое $r_k:B_{r_k}(a)\subset A_k$.

        $r:=\min\{r_1,...,r_k\}>0\Rightarrow B_r(a)\subset B_{r_k}(a)\subset A_k\ \forall k =1,..,n\Rightarrow B_r(a)\subset U$.

        \item[4.] $B_R(a)\ -$ открытое множество; $r:=R-\rho(a,x)$ и покажем, что $B_r(x)\subset B_R(a)$.

        (картиночка)

        Возьмем $y\in B_r(x)\Rightarrow \rho(x,y)<r\Rightarrow \rho(y,a)\leq \rho(y,x)+\rho(x,a)<r+\rho(x,a)=R$.
    \end{enumerate}
\end{proof}

\begin{definition}
    $A\subset X,\ a\in A;\ a\ -$ внутренняя точка $A$, если найдется $r>0:\ B_r(a)\subset A$.
\end{definition}

\begin{remark}
    $A\ -$ открытое множество $\Leftrightarrow$ всего его точки внутренние.
\end{remark}

\begin{definition}
    $\Int A\ -$ \textbf{внутренность множества} $A\ -$ множество всех внутренних точек.
\end{definition}

\begin{remark}
    Если $A\ -$ открытое множество, то $\Int A=A$.
\end{remark}


\begin{statement}
    \textbf{Свойства внутренности:}
    \begin{enumerate}
        \item $\Int A\ -$ объединение всех открытых множеств, содержащихся в $A$.
        \item $\Int A\ -$ открытое множество.
        \item $A\ -$открыто $\Leftrightarrow \Int A=A$.
        \item Если $A\subset B$, то $\Int A\subset \Int B$.
        \item $\Int (A\cap B)=\Int A\cap \Int B$.
        \item $\Int (\Int A)=\Int A$.
    \end{enumerate}
\end{statement}

\begin{proof}
    \begin{enumerate}
        \item[]
        \item[1.] $G:=\bigcup\limits_{U\in A,\ U\ -\text{ откр.}} $. Надо доказать, что $G=\Int A$.

        $\supset:$ Берем $a\in \Int A\Rightarrow \underset{\text{открытое}}{B_r(a})\subset A\Rightarrow a\in B_r(a)\subset G$.

        $\subset:$ Берем $a\in G\Rightarrow a\in U$ для некоторого открытого $U\subset A\Rightarrow B_r(a)\subset U\subset A\Rightarrow a\ -$ внутренняя точка $A\Rightarrow a\in \Int A$.
        \item[2.] Объединение открытых множеств $-$ открытое.
        \item[3.] $\Int A\ -$ открытое $\Rightarrow,\ \Leftarrow$ есть.
        \item[4.] Если $a\ -$ внутренняя точка $A$, то $B_r(a)\subset A\subset B\Rightarrow a\in \Int B$.
        \item[5.] $\begin{cases} A\cap B\subset A\Rightarrow \Int (A\cap B)\subset \Int A \\
        A\cap B\subset B\Rightarrow \Int (A\cap B)\subset \Int B\end{cases}\Rightarrow \subset$ есть.

        $\supset:$ Пусть $a\in \Int A\cap \Int B\Rightarrow \begin{cases}
                                                                B_{r_1}(a)\subset A \\ B_{r_2}(a)\subset B
        \end{cases} \Rightarrow B_{\min(r_1,r_2)}(a)\in A\cap B\Rightarrow a\in (\Int A\cap B)$.
        \item[6.] 2 + 3
    \end{enumerate}
\end{proof}

\begin{definition}
    $A\ -$ \textbf{замкнутое множество}, если $X\setminus A\ -$ открытое множество.
\end{definition}

\begin{theorem}
    \textbf{О свойствах замкнутых множеств:}
    \begin{enumerate}
        \item $\varnothing$ и $X\ -$ замкнутые множества.
        \item Пересечение любого числа замкнутых множеств $-$ замкнутое множество.
        \item Объединение конечного числа замкнутых множеств $-$ замкнутое множество.
        \item $\overline{B}_r\ -$ замкнутое множество.
    \end{enumerate}
\end{theorem}

\begin{proof}
    \begin{enumerate}
        \item[]
        \item Пусть $A_\alpha\ -$ замкнутое, $\alpha\in I$, $F:=\cap\limits_{\alpha\in I}A_\alpha$.

        Проверим, что $x\setminus F\ -$ открытое: $x\setminus F=x\setminus \cap\limits_{\alpha\in I}A_\alpha=\cup\limits_{\alpha\in I}\underset{\text{откр.}}{(X\setminus A_\alpha)}\ -$ открытое.
        \item Пусть $A_1,...,A_n\ -$ замкнутые, $F:=\cup\limits_{k=1}^nA_k$.

        Проверим, что $x\setminus F\ -$ открытое: $x\setminus F=x\setminus \cup\limits_{k=1}^nA_k=\cap\limits_{k=1}^n\underset{\text{откр.}}{(X\setminus A_k)}\ -$ открытое.
        \item Проверим, что $X\setminus \overline{B}_R(a)\ -$ открытое множество. Возьмем $x\notin \overline{B}_R(a)\Rightarrow \rho (a,x)>R$.

        (picture)

        $r:=\rho(x,a)-R>0$. Покажем, что $B_r(x)\subset X\setminus \overline{B}_R(a)$, т.е. что $B_r(x)\cap \overline{B}_R(a) =\varnothing$.

        Пусть $y\in B_r(x)\cap \overline{B}_R(a)\Rightarrow \rho(y,x)<r\And\rho(y, a)\leq R\Rightarrow \rho(a,y)+\rho(y,x)\leq R+r=\rho(a,x)$.
    \end{enumerate}
\end{proof}

\begin{remark}
    В третьем существенна конечность: $\bigcup\limits_{n=1}^\infty[\frac{1}{n}, 1]=(0,\frac{1}{n})$.
\end{remark}

\begin{definition}
    $\Cl A\ -$ \textbf{замыкание множества} $A\ -$ пересечение всех замкнутых множеств, содержащих $A$.
\end{definition}

\begin{theorem}
    $X\setminus \Cl A=\Int (X\setminus A)$.

    $X\setminus \Int A=\Cl (X\setminus A)$.
\end{theorem}

\begin{proof}
    $\Int(X\setminus A)=\cup\{U\ - $ открытое: $U\subset X\setminus A \}$

    $X\setminus \Int A=X \setminus \cup \{...\}=\cap \{X\setminus U: U\ - $ открытое: $U\subset X\setminus A \}=\cap \{ F : F\ - $ замкнутое и $\underset{\Leftrightarrow F \supset A}{X\setminus F \subset X\setminus A}$
\end{proof}

\begin{statement}
    \textbf{Свойства замыканий:}
    \begin{enumerate}
        \item $\Cl A\ -$ замкнутое множество.
        \item $A$ замкнуто $\Leftrightarrow \Cl A=A$.
        \item $A\subset B\Rightarrow \Cl A\subset \Cl B$.

        \underline{Комментарий:} $X\setminus A \supset X\setminus B \Rightarrow \Int (X\setminus A)\supset \Int (X\setminus B)$.
        \item $\Cl (A\cup B)=\Cl A\cup \Cl B$.

        \underline{Комментарий:} $X\setminus \Cl(A\cup B)=\Int (X\setminus (A\cup B))=\Int ((X\setminus A)\cap (X\setminus B))$.

        \item $\Cl (\Cl A)=\Cl A$.
    \end{enumerate}
\end{statement}