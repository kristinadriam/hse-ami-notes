\part{Метрические пространства}
\section{Метрические и нормированные пространства}

\begin{definition}
    \item $\rho:X\times X\rightarrow [0,+\infty)\ -$ \textit{метрика} (\textbf{расстояние}), если:
    \begin{enumerate}
        \item $\rho(x,x)=0$ и $\rho(x,y)=0\Rightarrow x=y$.
        \item $\rho(x,y)=\rho(y,x)\ \forall x,y\in X$.
        \item $\rho(x,y)\leq \rho(x,z) + \rho(z,y)\ \forall x,y,z\in X$ (неравенство треугольника).
    \end{enumerate}
\end{definition}

\begin{definition}
    Пара $(X,\rho)$ – это \textit{метрическое пространство}.
\end{definition}

\begin{example}
    \begin{enumerate}
        \item[]
        \item Дискретная метрика: $\rho(x,x)=0,\ \rho(x,y)=1$, если $x\neq y$
        \item $X=\R,\ \rho(x,y)=|x-y|$
        \item $X=\R^2$ расстояние на плоскости
        \item Манхэттеновская метрика: $X=\R^2,\ \rho((x_1, y_1), (x_2, y_2))=|x_1-x_2|+|y_1+y_2|$
        \item $\R^d=(x_1,...,x_d)$

        $\rho(x,y)=\sqrt{(x_1-y_1)^2+...+(x_d-y_d)^2}^*$ (на самом деле, даже $p$-ая степень и корень $p$-ой степени подойдет)

        $^*$Неравенство треугольника в этом случае $-$ это неравенство Минковского.
        \item $X=C[a,b],\ \rho(f,g)=\int\limits_a^b |f-g|$.
        \item Французская железнодорожная метрика: $\rho(A,B)=AB$, если $A$ и $B$ на одной прямой и $\rho(AB)=AP+PB$ иначе.

        (картиночка)
    \end{enumerate}
\end{example}

\begin{definition}
    $(X,\rho)$ – \textit{метрическое пространство}, $a\in X,\ r>0$.

    \textit{Открытый шар} $B_r(a):=\{x\in X\mid \rho(x,a)<r\}$.

    \textit{Замкнутый шар} $\overline{B}_r(a):=\{x\in X\mid \rho(x,a)\leq r\}$.

    $a$ – \textit{центр шара}, $r$ – \textit{радиус шара}.
\end{definition}

\begin{proper}
    \begin{enumerate}
        \item $B_{r_1}(a)\cap B_{r_2}(a)=B_{\min\{r_1, r_2\}}(a)$.
        \item Если $a\neq b$, то $\exists r>0:\overline{B}_r(a)\cap \overline{B}_r(b)=\varnothing$.
    \end{enumerate}
\end{proper}

\begin{proof}
    $r:=\frac{\rho(a,b)}{3}>0.$ Предположим, что $x\in \overline{B}_r(a)\cap \overline{B}_r(b)\Rightarrow \rho(x,a)\leq r\And \rho(x,b)\leq r\Rightarrow \rho(a,b)\leq \rho(a,x)+\rho(x,b)\leq r+r=\frac{2}{3}\rho(a,b)$, противоречие.
\end{proof}

\begin{definition}
    $A\subset X;\ A\ -$ \textit{открытое множество}, если $\forall a\in A$ найдется $B_r(a)\subset A$.
\end{definition}

\begin{theorem}
    \textbf{О свойствах открытых множеств}

    \begin{enumerate}
        \item $\varnothing$ и $X\ -$ открытые множества.

        \item Объединение любого числа открытых множеств $-$ открытое множество.

        \item Пересечение конечного числа открытых множеств $-$ открытое множество.

        \item $B_r(a)\ -$ открытое множество.
    \end{enumerate}
\end{theorem}

\begin{remark}
    В третьем конечность существенна: $\cap\limits_{n=1}^\infty (-1,\frac{1}{n})=(-1,0]$.
\end{remark}

\begin{proof}
    \begin{enumerate}
        \item[]
        \item[2.] $A_\alpha\ -$ открытые множества, $\alpha\in I$. Проверим, что $U:=\cup\limits_{n=1}A\ -$ открытое.

        Возьмем $a\in U\Rightarrow$ найдется $\alpha_0:a\in A_{\alpha_0}\ -$ открытое $\Rightarrow $ найдется $r>0:B_r(a)\subset A_\alpha \subset U$.

        \item[3.] $A_1,...,A_n\ -$ открытые множества. Проверим, что $U:=\cap\limits_{k=1}^n A_k\ -$ открытое.

        Возьмем $a\in U\Rightarrow a\in A_k\ k=1,..,n\ -$ открытое $\Rightarrow $ найдется такое $r_k:B_{r_k}(a)\subset A_k$.

        $r:=\min\{r_1,...,r_k\}>0\Rightarrow B_r(a)\subset B_{r_k}(a)\subset A_k\ \forall k =1,..,n\Rightarrow B_r(a)\subset U$.

        \item[4.] $B_R(a)\ -$ открытое множество; $r:=R-\rho(a,x)$ и покажем, что $B_r(x)\subset B_R(a)$.

        (картиночка)

        Возьмем $y\in B_r(x)\Rightarrow \rho(x,y)<r\Rightarrow \rho(y,a)\leq \rho(y,x)+\rho(x,a)<r+\rho(x,a)=R$.
    \end{enumerate}
\end{proof}

\begin{definition}
    $A\subset X,\ a\in A;\ a\ -$ \textit{внутренняя точка} $A$, если найдется $r>0:\ B_r(a)\subset A$.
\end{definition}

\begin{remark}
    $A\ -$ открытое множество $\Leftrightarrow$ всего его точки внутренние.
\end{remark}

\begin{definition}
    $\Int A\ -$ \textit{внутренность множества} $A\ -$ множество всех внутренних точек.
\end{definition}

\begin{remark}
    Если $A\ -$ открытое множество, то $\Int A=A$.
\end{remark}


\begin{proper}
    \textbf{Свойства внутренности:}
    \begin{enumerate}
        \item $\Int A\ -$ объединение всех открытых множеств, содержащихся в $A$.
        \item $\Int A\ -$ открытое множество.
        \item $A\ -$открыто $\Leftrightarrow \Int A=A$.
        \item Если $A\subset B$, то $\Int A\subset \Int B$.
        \item $\Int (A\cap B)=\Int A\cap \Int B$.
        \item $\Int (\Int A)=\Int A$.
    \end{enumerate}
\end{proper}

\begin{proof}
    \begin{enumerate}
        \item[]
        \item[1.] $G:=\bigcup\limits_{U\in A,\ U\ -\text{ откр.}} $. Надо доказать, что $G=\Int A$.

        $\supset:$ Берем $a\in \Int A\Rightarrow \underset{\text{открытое}}{B_r(a})\subset A\Rightarrow a\in B_r(a)\subset G$.

        $\subset:$ Берем $a\in G\Rightarrow a\in U$ для некоторого открытого $U\subset A\Rightarrow B_r(a)\subset U\subset A\Rightarrow a\ -$ внутренняя точка $A\Rightarrow a\in \Int A$.
        \item[2.] Объединение открытых множеств $-$ открытое.
        \item[3.] $\Int A\ -$ открытое $\Rightarrow,\ \Leftarrow$ есть.
        \item[4.] Если $a\ -$ внутренняя точка $A$, то $B_r(a)\subset A\subset B\Rightarrow a\in \Int B$.
        \item[5.] $\begin{cases} A\cap B\subset A\Rightarrow \Int (A\cap B)\subset \Int A \\
        A\cap B\subset B\Rightarrow \Int (A\cap B)\subset \Int B\end{cases}\Rightarrow \subset$ есть.

        $\supset:$ Пусть $a\in \Int A\cap \Int B\Rightarrow \begin{cases}
                                                                B_{r_1}(a)\subset A \\ B_{r_2}(a)\subset B
        \end{cases} \Rightarrow B_{\min(r_1,r_2)}(a)\in A\cap B\Rightarrow a\in (\Int A\cap B)$.
        \item[6.] 2 + 3
    \end{enumerate}
\end{proof}

\begin{definition}
    $A\ -$ \textit{замкнутое множество}, если $X\setminus A\ -$ открытое множество.
\end{definition}

\begin{theorem}
    \textbf{О свойствах замкнутых множеств:}
    \begin{enumerate}
        \item $\varnothing$ и $X\ -$ замкнутые множества.
        \item Пересечение любого числа замкнутых множеств $-$ замкнутое множество.
        \item Объединение конечного числа замкнутых множеств $-$ замкнутое множество.
        \item $\overline{B}_r\ -$ замкнутое множество.
    \end{enumerate}
\end{theorem}

\begin{proof}
    \begin{enumerate}
        \item[]
        \item Пусть $A_\alpha\ -$ замкнутое, $\alpha\in I$, $F:=\cap\limits_{\alpha\in I}A_\alpha$.

        Проверим, что $x\setminus F\ -$ открытое: $x\setminus F=x\setminus \cap\limits_{\alpha\in I}A_\alpha=\cup\limits_{\alpha\in I}\underset{\text{откр.}}{(X\setminus A_\alpha)}\ -$ открытое.
        \item Пусть $A_1,...,A_n\ -$ замкнутые, $F:=\cup\limits_{k=1}^nA_k$.

        Проверим, что $x\setminus F\ -$ открытое: $x\setminus F=x\setminus \cup\limits_{k=1}^nA_k=\cap\limits_{k=1}^n\underset{\text{откр.}}{(X\setminus A_k)}\ -$ открытое.
        \item Проверим, что $X\setminus \overline{B}_R(a)\ -$ открытое множество. Возьмем $x\notin \overline{B}_R(a)\Rightarrow \rho (a,x)>R$.

        (picture)

        $r:=\rho(x,a)-R>0$. Покажем, что $B_r(x)\subset X\setminus \overline{B}_R(a)$, т.е. что $B_r(x)\cap \overline{B}_R(a) =\varnothing$.

        Пусть $y\in B_r(x)\cap \overline{B}_R(a)\Rightarrow \rho(y,x)<r\And\rho(y, a)\leq R\Rightarrow \rho(a,y)+\rho(y,x)\leq R+r=\rho(a,x)$.
    \end{enumerate}
\end{proof}

\begin{remark}
    В третьем существенна конечность: $\bigcup\limits_{n=1}^\infty[\frac{1}{n}, 1]=(0,\frac{1}{n})$.
\end{remark}

\begin{definition}
    $\Cl A\ -$ \textit{замыкание множества} $A\ -$ пересечение всех замкнутых множеств, содержащих $A$.
\end{definition}

\begin{theorem}
    $X\setminus \Cl A=\Int (X\setminus A)$.

    $X\setminus \Int A=\Cl (X\setminus A)$.
\end{theorem}

\begin{proof}
    $\Int(X\setminus A)=\cup\{U\ - $ открытое: $U\subset X\setminus A \}$

    $X\setminus \Int A=X \setminus \cup \{...\}=\cap \{X\setminus U: U\ - $ открытое: $U\subset X\setminus A \}=\cap \{ F : F\ - $ замкнутое и $\underset{\Leftrightarrow F \supset A}{X\setminus F \subset X\setminus A}$
\end{proof}

\begin{proper}
    \textbf{Свойства замыканий:}
    \begin{enumerate}
        \item $\Cl A\ -$ замкнутое множество.
        \item $A$ замкнуто $\Leftrightarrow \Cl A=A$.
        \item $A\subset B\Rightarrow \Cl A\subset \Cl B$.

        \underline{Комментарий:} $X\setminus A \supset X\setminus B \Rightarrow \Int (X\setminus A)\supset \Int (X\setminus B)$.
        \item $\Cl (A\cup B)=\Cl A\cup \Cl B$.

        \underline{Комментарий:} $X\setminus \Cl(A\cup B)=\Int (X\setminus (A\cup B))=\Int ((X\setminus A)\cap (X\setminus B))$.

        \item $\Cl (\Cl A)=\Cl A$.
    \end{enumerate}
\end{proper}

\begin{theorem}
    $A$ – множество в метрическом пространстве $(X, \rho)$. Тогда $x\in \Cl A\Leftrightarrow \forall r>0$ B$_r(x)\cap A\neq \varnothing$.
\end{theorem}

\begin{proof}
    $\neg(P\Leftrightarrow Q)=\neg P\Leftrightarrow \neg Q$. Тогда докажем:

    $x\notin \Cl A\Leftrightarrow \exists r>0$ B$_r(x)\cap A=\varnothing$

    $x\notin \Cl A\Leftrightarrow x\in X\setminus \Cl A=\Int (X\setminus A)\Rightarrow x$ – внутренняя точка $X\setminus A\Rightarrow \exists r>0:$ $\underbrace{\text{B}_r(x)\subset X\setminus A}_{\text{B}_r(x)\cap A=\varnothing}$
\end{proof}

\begin{corollary}
    Если $U$ – открытое и $U\cap A=\varnothing$, то $U\cap \Cl A=\varnothing$.
\end{corollary}

\begin{proof}
    От противного. Пусть $U\cap \Cl A\neq\varnothing$. Возьмем $x\in U\cap \Cl A$.

    \begin{rcases}
        $x\in U$ – открытое $\Rightarrow\exists r>0:$ B$_r(x)\subset U$ \\
        $x\in A$ – замкнутое$\Rightarrow$ B$_r(x)\cap  A\neq \varnothing$
    \end{rcases} $\Rightarrow$ точка из B$_r(x)\cap A$ лежит и в $U$, и в $A$. Противоречие.
\end{proof}

\begin{definition}
    $U_a$ – \textit{окрестность точки} $a$  – шар B$_r(a)$ некоторого радиуса $r>0$.
\end{definition}

\begin{definition}
    $\overset{\circ}{U_a}$ – \textit{проколотая окрестность точки} $a$  – $\overset{\circ}{\text{B}}_r(a)=$B$_r(a)\setminus \{a\}$ некоторого радиуса $r>0$.
\end{definition}

\begin{definition}
    $a$ – \textit{предельная точка множества} $A$, если любая проколотая окрестность точки $a$ пересекается с множеством $A$.
\end{definition}

\begin{designation}
    $A'$ – множество всех предельных точек $A$.
\end{designation}

\begin{proper}
    \begin{enumerate}
        \item $\Cl A= A\cup A'$
        \begin{proof}
            $a\in \Cl A\Leftrightarrow \forall r> 0$ B$_r(a)\cap A\neq\varnothing\Leftrightarrow\left[\begin{gathered}
                                                                                                          a\in A \hfill \\
                                                                                                          \forall r > 0\ \overset{\circ}{\text{B}} _r(a)\Leftrightarrow a\in A'
            \end{gathered}\right.$
        \end{proof}
        \item $A\subset B\Rightarrow A'\subset B'$
        \item $A$ замкнуто $\Rightarrow A'\subset A$.

        \begin{proof}
            $A$ замкнуто $\Leftrightarrow A=\Cl A\Leftrightarrow A=A\cup A'\Leftrightarrow A'\subset A$.
        \end{proof}
        \item $(A\cup B)'=A'\cup B'$
        \begin{proof}
            $A\subset A\cup B\Rightarrow A'\subset (A\cup B)'\Rightarrow A'\cup B'\subset (A\cup B)'$

            Теперь докажем обратное включение. Возьмем $x\in (A\cup B)'\Rightarrow \forall r>0\ \overset{\circ}{\text{B}}_r(x)\cap (A\cup B)\neq \varnothing\ (1)$

            Пусть $x\notin B'\overset{\Rightarrow}{\text{для дост. малых } r}$B$_r(x)\cap B=\varnothing\ (2)$

            Из $(1)$ и $(2):\overset{\circ}{\text{B}}_r(x)\cap A\neq \varnothing$ для достаточно малых $r\Rightarrow\forall r>0\ \overset{\circ}{\text{B}}_r(x)\cap A\neq \varnothing$
        \end{proof}
    \end{enumerate}
\end{proper}

\begin{theorem}
    $x\in A'\Leftrightarrow\exists r>0:$ шар B$_r(x)$ содержит бесконечно много точек из $A$.
\end{theorem}

\begin{proof}
    $\Leftarrow:$ если B$_r(x)\cap A$ содержит бесконечно много точек, то $\overset{\circ}{\text{B}_r}(x)\cap A$ содержит  бесконечно много точек $\Rightarrow\overset{\circ}{\text{B}_r}(x)\cap A\neq \varnothing\Rightarrow x\in A'$.

    $\Rightarrow:x\in A'\Rightarrow \overset{\circ}{\text{B}_1}(x)\cap A\neq \varnothing$. Возьмем точку $x_1$ из этого пересечения, для нее $0<\underbrace{\rho(x,x_1)}_{=:r_2}<1$

    $\overset{\circ}{\text{B}_{r_2}}(x)\cap A\neq \varnothing$. Возьмем точку $x_2$ из этого пересечения, для нее $0<\underbrace{\rho(x,x_2)}_{=:r_3}<r_2=\rho(x,x_1)$.

    И так далее: $\rho(x,x_1)>\rho(x, x_2)>...\Rightarrow$ все радиусы различны.

\end{proof}

\begin{remark}
    Можно добиться того, что $\rho(x, x_n)\rightarrow 0$. Для этого надо брать $r_n=\frac{1}{2}\rho(x,x_{n-1})$.
\end{remark}

\begin{corollary}
    Конечное множество не имеет предельных точек.
\end{corollary}

\begin{definition}
    Пусть $(X,\rho)$ – метрическое пространство, $Y\subset X$. \textit{Подпространством метрического пространства} называется $(Y, \rho|_{Y\times Y})$ (более кратко: $(Y, \rho)$).
\end{definition}

\begin{theorem}
    \textbf{Об открытых и замкнутых множествах в подпространстве}

    Пусть $(X, \rho)$ – метрическое пространство $A\subset Y\subset X$. Тогда:
    \begin{enumerate}
        \item $A$ открыто в $(Y, \rho)\Leftrightarrow \exists G$ – открытое в $(X,\rho):\ A=G\cap Y$.
        \item $A$ замкнутое в $(Y, \rho)\Leftrightarrow \exists F$ – замкнутое в $(X,\rho):\ A=F\cap Y$.
    \end{enumerate}
\end{theorem}

\begin{proof}
    \begin{enumerate}
        \item $\Rightarrow:A$ открытое в $Y\Rightarrow A=\bigcup\limits_{x\in A}$B$_{r_x}^Y(x)$, где $r_x>0$ такой радиус, что B$_{r_x}\subset A$.

        B$_{r_x}^Y=\{y\in Y:\rho(x,y)<r_x\}=\underbrace{\{y\in X:\rho(x,y)<r_x\}}_{=\text{B}^X_{r_x}(x)}\cap Y=$B$_{r_x}^X\cap Y\Rightarrow A = \bigcup\limits_{x\in A}($B$_{r_x}^X(x)\cap Y)=Y\cap \underbrace{\cup\text{B}^X_{r_x}}_{=:G\text{ откр. в }X}$

        $\Leftarrow:A=Y\cap G$, $G$ – открытое в $X\Rightarrow G=\bigcup\limits_{x\in G}\text{B}^X_{r_x}(x),$ где $r_x>0$ такой радиус, что B$_{r_x}\subset G\Rightarrow A = Y\cap \bigcup\limits_{x\in G}\text{B}^X_{r_x}(x)=\bigcup\limits_{x\in G}(Y\cap \text{B}^X_{r_x}(x))=\bigcup\limits_{x\in G}\text{B}^Y_{r_x}(x)$ –  открытое в $Y$.

        \item $A$ замкнуто в $Y\Leftrightarrow Y\setminus A$ открыто в $Y\overset{1}{\Leftrightarrow}\exists G$ – открытое в $X:Y\setminus A=G\cap Y\Leftrightarrow A=Y\setminus G=Y\cap(\underset{\text{замк. }F:=}{}{X\setminus G})\Leftrightarrow \exists F$ – замкнутое в $X:A=Y\cap F$ (и обратно доказывается с конца).
    \end{enumerate}
\end{proof}

\begin{example}
    $x=\R,\ \rho(x,y):=|x-y|,\ Y=[0,3)$

    В $(Y, \rho)$ :

    $[0,1)$ – открытое множество $[0,1)=(-1,1)\cap Y$.

    $[2,3)$ – закрытое множество $[2,3)=[2,4]\cap Y$.
\end{example}

\begin{definition}
    $X$ – векторное пространство над $\R$. $\|.\|:X\rightarrow \R$ – \textit{норма}, если:
    \begin{enumerate}
        \item $\|x\|\leq 0$ и $\|x\|=0\Leftrightarrow x=\vec{0}$.
        \item $\|\lambda x\|=|\lambda|\cdot \|x\|,\ \forall\lambda\in \R,\  \forall x,y,z$.
        \item $\|x+y\|\leq \|x\|+ \|y\|\ \forall x,y\in X$ (неравенство треугольника).
    \end{enumerate}
\end{definition}

\begin{example}
    \begin{enumerate}
        \item[]
        \item $|x|$ в $\R$
        \item $\|x\|_p:=(|x_1|^p+...+|x_d|^p)^{\frac{1}{p}}$ в $\R^d$ при $p\geq 1$.
        \item $\|x\|_\infty:=\max\{|x_1|, ..., |x_d|\}$ в $\R^d$.
        \item $X=C[a,b];\ \|f\|:=\underset{x\in[a,b]}{\max}|f(x)|$.
        \item $X=C[a,b];\ \|f\|:=\int\limits_a^b|f(x)|dx$.
    \end{enumerate}
\end{example}

\begin{definition}
    $X$ – векторное пространство над $\R$. $\langle \ ,\ \rangle:X\times X\rightarrow \R$

    \textit{скалярное произведение}, если:
    \begin{enumerate}
        \item $\langle x,x \rangle\leq 0$ и $\langle x,x\rangle=0\Leftrightarrow x=\vec{0}$.
        \item $\langle x+y, z\rangle=\langle x,z \rangle +\langle y, z\rangle\ \forall x,y,z$.
        \item $\langle\alpha x, y \rangle=\lambda\langle x,y \rangle\ \forall x,y\in X$ и $\forall\lambda\in \R$.
        \item $\langle x,y\rangle=\langle y,x\rangle\ \forall x,y\in X$.
    \end{enumerate}
\end{definition}

\begin{example}
    \begin{enumerate}
        \item[]
        \item $X=\R^d;\ \langle x,y\rangle=\sum_{k=1}^d x_ky_k$
        \item $X=\R^d,\ w_1,...,w_d>0;\ \langle x,y\rangle=\sum\limits_{k=1}^d w_k x_k y_k$
        \item $X\in C[a,b];\ \langle f,g\rangle=\int\limits_b^af(x)g(x)dx$.
    \end{enumerate}
\end{example}

\begin{proper}
    \begin{enumerate}
        \item[]
        \item \textbf{Неравенство Коши-Буняковского}: $\langle x,y\rangle^2\leq \langle x,x\rangle\langle y,y\rangle$
        \begin{proof}
            Пусть $y\neq \vec{0}$ (если $y=\vec{0}$, то очевидно).

            $f(t):=\langle x+ty,x+ty\rangle=\langle x,x\rangle+t\langle y,x\rangle+t\langle x,y\rangle+t^2\langle y,y\rangle=\langle t,y\rangle t^2+2\langle x,y\rangle+\langle x,x\rangle$ – квадратный трехчлен.

            $f(t)\geq 0\ \forall t\in R\Rightarrow$ его дискриминант $\leq 0\Rightarrow (2\langle x,y\rangle)^2-4\langle x,x\rangle\langle y,y\rangle\leq 0\Leftrightarrow \langle x,y\rangle^2\leq \langle x,x\rangle\langle y,y\rangle$
        \end{proof}
        \item $\|x\|:=\sqrt{\langle x,x\rangle}$ – норма.
        \begin{proof}
            \begin{enumerate}
                \item[]
                \item $\|x\|\geq 0$ и $\|x\|=0\Leftrightarrow x=\vec{0}$ - первое свойство скалярного произведения.
                \item $\|\lambda x\|=\sqrt{\langle\lambda x,\lambda x\rangle}=\sqrt{\lambda^2\langle x,x\rangle}=|\lambda|\sqrt{\langle x,x\rangle}=|\lambda|\cdot \|x\|$
                \item $\|x+y\|\leq \|x\|+\|y\|$

                $\|x+y\|^2\leq \|x\|^2+2\|x\|\cdot \|y\|+\|y\|^2$

                $\langle x,x\rangle+2\langle x,x
                y\rangle+\langle y,y\rangle=\langle x+y,x+y\rangle\leq \langle x,x\rangle+\langle y,y\rangle+2\sqrt{\langle x,x\rangle\langle y,y\rangle}$ и осталось неравенство Коши-Буняковского.
            \end{enumerate}
        \end{proof}
        \item $\rho(x,y)=\|x-y\|$ – метрика.
        \begin{proof}
            $\rho(y,x)=\|y-x\|=\|(-1)(x-y)\|=|-1|\|x-y\|=\|x-y\|=\rho(x,y)$

            Неравенство треугольника: $\rho(x,y)+\rho(y,z)\geq \rho(x,z)$

            $\|x-y\|+\|y-z\|\geq \|(x-y)+(y-z)\|=\|x-z\|$
        \end{proof}
        \item $\|x-y\|\geq |\|x\|-\|y\||$

        \begin{proof}
            $\|x\|-\|y\|\leq \|x-y\|$ и $\|y\|-\|x\|\leq \|y-x\|=\|x-y\|$

            $\|x\|=\|y+(x-y)\|\leq \|y\|+\|x-y\|$
        \end{proof}
    \end{enumerate}
\end{proper}

\begin{definition}
    $(X,\rho)$ – метрическое пространство, $x_n$ – последовательность в $X,\ a\in X$. Тогда $\lim x_n = a$:
    \begin{enumerate}
        \item Вне любого шара B$_r(a)$ содержится лишь конечное число членов последовательности.
        \item $\forall \varepsilon > 0\ \exists N:\ n\geq N\ \rho(x_n, a)<\varepsilon$.
    \end{enumerate}
\end{definition}

\begin{proper}
    \begin{enumerate}
        \item[]
        \item $\lim x_n=a\Leftrightarrow \lim \rho(x_n,a)=0$.
        \item Предел единственнен.
        \begin{proof}
            Если $\lim x_n=a$ и $\lim x_n=b$, то возьмем такое число $r>0$, что B$_r(a)\cap$B$_r(b)\neq \varnothing\Rightarrow$ вне B$_r(a)$ конечное число членов и вне B$_r(b)$ конечное число членов $\Rightarrow$ всего конечное число членов. Противоречие.
        \end{proof}
        \item Если $\lim x_n=a$ и $\lim y_n=a$, то последовательность, полученная перемешиванием $x_n$ и $y_n$, также стремится к $a$.
        \item Если $\lim x_n=a$, то последовательность, полученная перестановкой членов последовательности, имеет тот же предел.
        \item Если $\lim x_n=a$, то последовательность, в которой $x_n$ взяты с конечной кратностью, имеет тот же предел.

        \begin{definition}
            $A\subset X;$ $A\ -$ \textit{ограниченное множество}, если $A$ содержится в некотором шаре.
        \end{definition}

        \item Если последовательность имеет предел, то она ограничена.
        \begin{proof}
            Возьмем $\varepsilon=1$. $\exists N:\ \forall n\geq N \ \rho(x_n, a)<1$

            $R:=\{\rho(x_1,a),...,\rho(x_{N-1}, a)\}+1\Rightarrow x_n\in$B$_r(a)$.
        \end{proof}
        \item Если $\lim x_n=a$, то $\lim x_{n_k}=1$.
        \item $a$ – предельная точка множества $A\Leftrightarrow$ существует последовательность $\underset{\neq a}{x_n}\in A$: $\lim x_n=a$. Более того, $x_n$ можно выбрать так, что $\rho(x_n,a)$ монотонно убывают.
        \begin{proof}
            $\Rightarrow:$ было.

            $\Leftarrow:$ берем B$_r(a)\Rightarrow\exists N:\ \forall n\geq N\ \underset{\neq a}{x_n}\in$B$_r(a)\Rightarrow\overset{\circ}{\text{B}}_r(a)\cap A\neq \varnothing$
        \end{proof}
    \end{enumerate}
\end{proper}

\begin{theorem}
    \textbf{Об арифметических действиях с пределами}

    $X$ – векторное пространство, $\|\cdot\|$ – норма в $X$, $\lim x_n=a,\ \lim y_n=b,\ \lim \lambda_n=\nu;\ \lambda_n,\nu\in \R;\ x_n,y_n,a,b\in X$. Тогда:
    \begin{enumerate}
        \item $\lim (x_n \pm y_n) = a\pm b$.
        \item $\lim (\lambda_n x_n) = \nu a$.
        \item $\lim \|x_n\|=\|a\|$.
        \item Если в $X$ есть скалярное произведение, то $\lim \langle x_n, y_n\rangle =\langle a,b\rangle$.
    \end{enumerate}
    \begin{proof}
        \begin{enumerate}
            \item[]
            \item $\rho(x_n+y_n, a+b)=\|(x_n+y_n)-(a+b)\|\leq \|x_n-a\|+\|y_n-b\|=\underset{\rightarrow 0}{\rho(x_n-a)}+\underset{\rightarrow 0}{\rho(y_n-b)}\Rightarrow \rho(x_n+y_n, a+b)\rightarrow 0$
            \item $\|\lambda_nx_n-\nu a\|=\|\lambda_nx_n-\lambda_na+\lambda_na-\nu a\|\leq \|\lambda_nx_n-\lambda_n a\|+\|\lambda_n a-\nu a\|=\underset{\text{огр.}}{|\lambda_n|}\underset{\rightarrow 0}{\|x_n-a\|}+\underset{\rightarrow 0}{|\lambda_n-\nu|}\|a\|\rightarrow 0$
            \item $|\|x_n\|-\|a\||\leq\|x_n\|-\|a\|\rightarrow0$
            \item $\langle x,y\rangle=\frac{1}{4}(\|x+y\|^2-\|x-y\|^2)$

            $\langle x_n,y_n\rangle=\frac{1}{4}(\underset{\rightarrow\|a+b\|^2}{\|x_n+y_n\|^2}-\underset{\rightarrow\|a-b\|^2}{\|x_n-y_n\|^2})\rightarrow \frac{1}{4}(\|a+b\|^2-\|a-b\|^2)=\frac{1}{4}\langle a,b\rangle$
        \end{enumerate}
    \end{proof}
\end{theorem}
