\section{Несобственные интегралы}

\begin{definition}
    $-\infty < a < b\leq +\infty,\ f\in C[a,b).$ Тогда \textbf{несобственный интеграл}:

    $\int\limits_a^{\rightarrow b}f:=\lim \limits_{B\rightarrow b_-}\int\limits_a^Bf$, если предел существует.
\end{definition}

\begin{definition}
    $-\infty \leq a < b< +\infty,\ f\in C(a,b].$ Тогда \textbf{несобственный интеграл}:

    $\int\limits_{\rightarrow a}^b f:=\lim \limits_{A\rightarrow a_+}\int\limits_A^b f$, если предел существует.
\end{definition}

\begin{definition}
    Если предел существует и конечен, то соответствующий интеграл назовем \textbf{сходящимся}. В остальных случаях назовем интеграл \textbf{расходящимся}.
\end{definition}

\begin{remark}
    \begin{enumerate}
        \item[]
        \item Если $b\neq -\infty$ и $f\in C[a,b]$, то $\int\limits_a^{\rightarrow b} f = \int\limits_a^{b} f$.

        \underline{Комментарий:} $\int\limits_a^{\rightarrow b} f=\lim\limits_{B\rightarrow b_-}\int\limits_a^{b}f,\ \bigg|\int\limits_a^b f-\int\limits_a^B\bigg|=\bigg|\int\limits_B^b\bigg|\leq (b-B)M$, где $M\ -\max |f|$.

        \item Если $f$ имеет первообразную $F$ в $[a,b)$, то $\int\limits_a^{\rightarrow b}=\lim\limits_{B\rightarrow b_-} F(b)-F(a)$.

        \underline{Комментарий:} $\int\limits_b^B=F(b)-F(a)$ и написать пределы.
    \end{enumerate}
\end{remark}

\begin{theorem}
    \textbf{Критерий Коши для несобственных интегралов}

    $-\infty <a<b\leq +\infty,\ f\in C[a,b)$; тогда $\int\limits_a^{\rightarrow b}f$ сходится $\Leftrightarrow\forall \varepsilon>0\ \exists C\in (a,b): $

    $\forall A,B\in (c,b) \bigg|\int\limits_A^B f\bigg |<\varepsilon$.
\end{theorem}

\begin{proof}
    Пусть $F:[a,b)\rightarrow \R\ -$ первообразная $f$. Тогда $\int\limits_a^{\rightarrow b}f$ сходится $\Leftrightarrow \exists$ конечный $\lim\limits_{B\rightarrow b_-}F(b)$.

    Если $b\neq +\infty \Leftrightarrow \forall \varepsilon >0\ \exists \delta > 0\ \forall A,B\in (\underbrace{b-\delta}_{=c};b)\ \underbrace{|F(A)-F(B)|}_{=\int\limits_A^B}<\varepsilon$

    Если $b= +\infty \Leftrightarrow \forall \varepsilon >0\ \exists E\ \forall A,B \supset \underbrace{E}_{=c}\ \underbrace{|F(A)-F(B)|}_{=\int\limits_A^B}<\varepsilon$
\end{proof}

\begin{remark}
    Если $\exists A_n,\ B_n\in [a,b)$, т.ч. $A_n,\ B_n\rightarrow b$ и $\int\limits_{\underbrace{A_n}_{:=C_n}^{B_n}}\nrightarrow 0$, то $\int\limits_a^{\rightarrow b}f$ расходится.

    Найдется подпоследовательность $C_{n_k}$, т.ч. $|C_{n_k}|>\varepsilon\Rightarrow \bigg|\int\limits_{B_n}^{A_n}f\bigg|\geq \varepsilon$ ?! (противоречие с критерием Коши).
\end{remark}

\begin{example}
    \begin{enumerate}
        \item[]
        \item $\int\limits_1^{+\infty }\frac{dx}{x^p}=\lim\limits_{x\rightarrow +\infty}F(x)-F(1)$, где $F(x)\ -$ первообразная $\frac{1}{x^p}$.

        Если $p=1$, то $F(x)=\ln x$ и $\lim\limits_{x\rightarrow +\infty}\ln x=+\infty\Rightarrow$ интеграл расходится.

        Если $p\neq1$, то $F(x)=-\frac{1}{(p-1)\cdot x^{p-1}}$ и $\lim\limits_{x\rightarrow +\infty}\frac{1}{x^{p-1}}=\begin{cases}
                                                                                                                          0, & \text{если $p>1\Rightarrow$ интеграл сходится} \\
                                                                                                                          +\infty & \text{если $p<1\Rightarrow$ интеграл расходится}
        \end{cases}$

        \underline{Итог:} $\int\limits_1^{+\infty }\frac{dx}{x^p}$ сходится $\Leftrightarrow p>1$ и в этом случае $\int\limits_1^{+\infty }\frac{dx}{x^p}=\frac{1}{p-1}$.
        \item $\int\limits_0^1\frac{dx}{x^p}=F(1)-\lim\limits_{x\rightarrow 0_+}F(x)$, где $F(x)\ -$ первообразная $\frac{1}{x^p}$.

        Если $p=1$, то $F(x)=\ln x$ и $\lim\limits_{x\rightarrow 0_+}\ln x=-\infty\Rightarrow$ интеграл расходится.

        Если $p\neq1$, то $F(x)=-\frac{1}{(p-1)\cdot x^{p-1}}$ и $\lim\limits_{x\rightarrow 0_+}\frac{1}{x^{p-1}}=\lim\limits_{x\rightarrow 0_+}x^{1-p}=\begin{cases}
                                                                                                                                                            0, & \text{если $p<1\Rightarrow$ интеграл сходится} \\
                                                                                                                                                            +\infty & \text{если $p>1\Rightarrow$ интеграл расходится}
        \end{cases}$

        \underline{Итог:} $\int\limits_0^1\frac{dx}{x^p}$ сходится $\Leftrightarrow p<1$ и в этом случае $\int\limits_0^1\frac{dx}{x^p}=\frac{1}{1-p}$.
    \end{enumerate}
\end{example}

\begin{definition}
    $f$ непрерывно на $[a,b]$ за исключением точек $c_1,...,c_n$.

    Рассмотрим $\int\limits_a^{d_1} f,\ \int\limits_{d_1}^{c_1} f,\ \int\limits_{c_1}^{d_2} f,..., \int\limits_{d_{n+1}}^{b} f$.

    Если все интегралы сходятся, то $\int\limits_a^b f$ сходится и $\int\limits_a^b f=\int\limits_a^{d_1} f+\int\limits_{d_1}^{c_1} f+ \int\limits_{c_1}^{d_2} f+...+ \int\limits_{d_{n+1}}^{b} f$.

    В противном случае интеграл расходится.
\end{definition}

\begin{statement}
    \textbf{Свойства несобственных интегралов:}
    \begin{enumerate}
        \item \textbf{Аддитивность}

        $c\in (a,b)$; если $\int\limits_a^b f$ сходится, то $\int\limits_c^b f$ сходится и $\int\limits_a^b f=\int\limits_a^c f+\int\limits_c^b f$.

        \begin{proof}
            $F\ -$ первообразная $f$; $\int\limits_a^b f = \lim\limits_{B\rightarrow b_-}F(b)-F(a)$

            Сходимость $\int\limits_a^b f\Leftrightarrow \lim\limits_{B\rightarrow b_-}F(b)$ существует и конечен.

            $\int\limits_b^c f=\lim\limits_{B\rightarrow b_-}F(b)-F(c)=\int\limits_a^b f - \underbrace{(F(c)-F(a))}_{\int\limits_a^c f}$
        \end{proof}
        \item \textbf{Линейность}

        Если $\int\limits_a^b f$ и $\int\limits_a^b g$ сходятся, $\alpha, \beta\in \R$, то $\int\limits_a^b(\alpha f+\beta g)$ сходится и $\int\limits_a^b(\alpha f+\beta g)=\alpha\int\limits_a^b f\beta \int\limits_a^b g$.
        \begin{proof}
            $F$ и $G\ -$ первообразные для $f$ и $g$; по условию $\lim\limits_{B\rightarrow b_-} F(B)$ и $\lim\limits_{B\rightarrow b_-} G(B)$ существуют и конечные $\Rightarrow\alpha F + \beta G\ -$ первообразные для $\alpha f + \beta g$ и $\lim\limits_{B\rightarrow b_-}(\alpha\cdot F(B) + \beta\cdot G(B))=\alpha\lim\limits_{B\rightarrow b_-}F(B) + \beta\lim\limits_{B\rightarrow b_-} G(B)\Rightarrow \int\limits_a^b (\alpha f + \beta g)$ сходится

            и $\int\limits_a^b (\alpha f + \beta g)=\alpha\lim\limits_{B\rightarrow b_-}F(B) + \beta\lim\limits_{B\rightarrow b_-} G(B)-\alpha \cdot F(A) - \beta \cdot G(A)=\alpha \cdot \int\limits_a^b f+\beta \cdot \int\limits_a^b g$
        \end{proof}
        \begin{remark}
            Если $\int\limits_a^b f$ сходится и $\int\limits_a^b g$ расходится, то $\int\limits_a^b(f+g)$ расходится.
        \end{remark}

        Комментарий: $g=(f+g)-f$
        \item \textbf{Монотонность}
        Если $\int\limits_a^b f$ и $\int\limits_a^b g$ существуют в $\overline{\R}$ и $f\leq g$ во всех точках от $a$ до $b$, то $\int\limits_a^b f \leq \int\limits_a^b g$.
        \begin{proof}
            $\int\limits_a^B f\leq \int\limits_a^B g$ и перейти к пределу.
        \end{proof}
        \item \textbf{Формула интегрирования по частям}

        Если $f, g\in C^{1}[a,b)$, то $\int\limits_a^b fg' = fg\left.\right|_a^{b^{\leftarrow\text{тут предел}}}-\int\limits_a^bf'g$.

        Если существует два конечных предела, то существует и третий и есть равенство.
        \begin{proof}
            $\int\limits_a^B fg' = fg\left.\right|_a^B-\int\limits_a^Bf'g$ и перейти к пределу.
        \end{proof}
        \item \textbf{Замена переменной}

        $\varphi: [\alpha, \beta)\rightarrow [a, b),\ \varphi\in C^{-1}[\alpha, \beta),\ \exists \lim\limits_{\gamma \rightarrow \beta_-}\varphi(\gamma)=:\varphi(\beta_-),\ f\in C[a,b)$, тогда:

        $\int\limits_\alpha^\beta f(\varphi(t))\varphi'(t)dt=\int\limits_{\varphi(\alpha)}^{\varphi(\beta_-)}f(x)dx$ (если существует один $\int$, то существует и другой и они равны).

        \begin{proof}
            $F(y):=\int\limits_{\varphi(x)}^y f(x)dx,\ \Phi(\gamma):=\int\limits f(\varphi(t))\varphi'(t)dt,\ Phi(\gamma)=F(\varphi(y))$ при $\alpha<\gamma<\beta$.

            Далее рассмотрим следующие случаи:
            \begin{enumerate}
                \item[I.] Если $\exists \lim\limits_{y\rightarrow \varphi (\beta_-)} F(y)$.

                Возьмем $\gamma_n\uparrow \beta\Rightarrow \varphi(\gamma_n)\rightarrow \varphi(\beta_-)\Rightarrow \int\limits_{\alpha}^{\gamma_n}f(\varphi(t))\varphi'(t)dt=\Phi (\gamma_n)=F\varphi(\varphi(\gamma_n))=\lim\limits_{y\rightarrow \varphi(\beta_-)} F(y)\int\limits_{\varphi(\alpha)}^{\varphi(\beta_-)}f(x)dx$

                \item[II.] Если $\exists \lim\limits_{\gamma\rightarrow \beta_-} \Phi(\gamma)$.

                Проверим, что $\exists \lim\limits_{y\rightarrow \varphi(\beta_-)} F(y)$.

                При $\varphi(\beta_-)<b$ очевидно, поскольку $F\in C[a,b)$. Пусть $\varphi(\beta_-)=b$. Возьмем $b_n\uparrow b$. Считаем, что $b_n\in [\varphi(\alpha),b)$. Тогда $\exists \gamma_n\in [\alpha, \beta)$ т.ч. $\varphi(\gamma_n)=b)n$.

                Докажем, что $\gamma_n\rightarrow \beta$.

                От противного. Найдется $\gamma_{n_k}\rightarrow \Tilde{\beta}<\beta\Rightarrow b_{n_k}=\varphi(\gamma_n)\rightarrow \varphi(\Tilde{\beta})<b\ (\varphi$ непрерывна в $\Tilde{\beta})$. Противоречие с тем, что $b_n\rightarrow b$. Следовательно $\gamma_n\rightarrow \beta$.

                $F(b_n)=F(\varphi(\gamma_n))=\Phi(\gamma_n)$ имеет предел $\overset{\text{по Гейне}}{\Rightarrow}\exists \lim\limits_{y\rightarrow b_-}F(y)$
            \end{enumerate}
        \end{proof}
    \end{enumerate}
\end{statement}

\begin{remark}
    $\int\limits_a^b f$ заменой $x=b-\frac{1}{t}$ сводится к $\int\limits_{\frac{1}{b-a}}^{+\infty}f(b-\frac{1}{t})\frac{1}{t^2}dt$.
\end{remark}

\begin{theorem}
    Пусть $f\in C[a,b]$ и $f\geq 0$. Тогда сходимость $\int\limits_a^b f$ равносильна ограниченности сверху функции $F(y):=\int\limits_a^y$.
\end{theorem}

\begin{proof}
    Если $f\geq 0$, то $F\ -$ возрастающая функция: $F(y)-F(x)=\int\limits_x^y f\geq 0$

    $\int\limits_a^b f\ -$ сходится $\Leftrightarrow \lim\limits_{y\rightarrow b_-} F(y)$ существует и конечен, а так как $F$ возрастает, то это равносильно ограниченности $F$ сверху.
\end{proof}

\begin{corollary}
    \textbf{Признак сравнения}
    $f, g\in C[a,b],\ f,g\geq 0$ и $f\leq g$, тогда:

    \begin{enumerate}
        \item Если $\int\limits_a^b g$ сходится, то $\int\limits_a^b f$ сходится.

        \item Если $\int\limits_a^b f$ расходится, то $\int\limits_a^b g$ расходится.
    \end{enumerate}
\end{corollary}

\begin{proof}
    $F$ и $G$ первообразные. Знаем, что $F(x)\leq G(x)$: $F(x)=\int\limits_a^x f \leq \int\limits_a^x g=G(x)$.

    Если $\int\limits_a^b g$ сходится, то $G$ ограничена сверху $\Rightarrow F$ ограничена сверху $\overset{\text{по th}}{\Rightarrow}\int\limits_a^b f$ сходится.

    Второй пункт = отрицание первого.
\end{proof}

\begin{remark}
    \begin{enumerate}
        \item[]
        \item Неравенство $f\leq g$ может выполняться лишь при аргументах, близких к $b$.
        \item Неравенство $f\leq g$ можно заменить на $f=o(g)$.
        \item Если $f\in C[a,+\infty),\ f=\mathcal{O}(\frac{1}{x^{1+\varepsilon}})$ при $\varepsilon>0$, то $\int\limits_a^{+\infty} f$ сходящийся.
    \end{enumerate}
\end{remark}

\begin{corollary}
    Пусть $f,g\in C[a,b),\ f,g\geq 0$ и $f(x)\sim g(x)$ при $x\rightarrow b_-$. Тогда $\int\limits_a^b f$ и $\int\limits_a^b g$ ведут себя одинаково (либо оба сходятся, либо оба расходятся).
\end{corollary}

\begin{proof}
    $f(x)=\varphi(x)g(x),$ где $\varphi(x)\rightarrow 1\Rightarrow$ при $x$ близких к $b$; $\frac{1}{2}\leq \varphi(x)\leq 2\Rightarrow \begin{cases}
                                                                                                                                            f(x) \leq 2 g(x) & \text{при $x$ близких к $b\Rightarrow$ если $\int\limits_a^b g$ сходящийся, то и $\int\limits_a^b f$ сходящийся} \\
                                                                                                                                            g(x) \leq 2 f(x) & \text{при $x$ близких к $b\Rightarrow$ если $\int\limits_a^b f$ сходящийся, то и $\int\limits_a^b g$ сходящийся} \\
    \end{cases}$
\end{proof}

\begin{remark}
    Если $\int\limits_a^{+\infty}f$ сходящийся и $f\geq 0$, то необязательно, что $\lim\limits_{x\rightarrow +\infty} f(x)=0$.

    $\frac{1}{2}\sum\limits_{n=0}^\infty\frac{1}{2^n}=1$
\end{remark}

\begin{definition}
    $f:[a,b]\rightarrow \R$ непрерывна. $\int\limits_a^b f$ называется \textit{абсолютно сходящимся}, если $\int\limits_a^b |f|<+\infty$.
\end{definition}

\begin{theorem}
    Если $\int\limits_a^b f$ абсолютно сходящийся, то он сходится.
\end{theorem}

\begin{proof}
    $|f|=f_++f_-,\ f_\pm \geq 0\Rightarrow 0\leq f_\pm\leq |f|\Rightarrow \int\limits_a^b f_\pm\ -$ сходящийся $\Rightarrow \int\limits_a^b f=\int\limits_a^b f_++\int\limits_a^b f_-\ -$ сходящийся.
\end{proof}

\begin{theorem}
    \textbf{Признак Дирихле}

    $f,g\in C[a,+\infty)$
    $\begin{cases}
         1)\ f\text{ имеет первообразную.} \\
         2)\ g\text{  монотонна.} \\
         3)\lim\limits_{x\rightarrow +\infty} g(x)=0
    \end{cases}$ $\Rightarrow \int\limits_a^{+\infty} f(x)g(x)dx$ сходится.
\end{theorem}

\begin{proof}
    \textit{Только для $g\in C^1[a,+\infty$}.

    $F(y):=\int\limits_a^y f(x)g(x)dx\ -$ ограниченная функция.

    $\int\limits_a^y f(x)g(x)dx = \int\limits_a^y F'(x)g(x)dx = \left. F(x)g(x)\right\bigg|_a^y -int\limits_a^{+\infty}F(x)g'(x)dx$

    $F(y)g(y)\underset{y\rightarrow +\infty}{\rightarrow} 0$ (ограниченная на бесконечно малую) $\Rightarrow$ надо доказать, что $\int\limits_a^{+\infty}F'(x)g(x)dx$ сходится. Докажем, что он абсолютно сходящийся.

    $\int\limits_a^{+\infty}|F'(x)||g(x)|dx\leq M\int\limits_a^{+\infty}|g(x)|dx=M\bigg|\int\limits_a^{+\infty}g(x)dx\bigg|=M|\left.g(x)\right|_a^{+\infty}|=M|g(a)|<+\infty$
\end{proof}

\begin{theorem}
    \textbf{Признак Абеля}

    $f,g\in C[a,+\infty)$
    $\begin{cases}
         1)\ \int\limits_a^{+\infty} f(x)dx\text{ сходится} \\
         2)\ g\text{  монотонна.} \\
         2)\ g\text{  ограничена.}
    \end{cases}$ $\Rightarrow \int\limits_a^{+\infty} f(x)g(x)dx$ сходится.
\end{theorem}

\begin{proof}
    Пусть $b:=\lim\limits_{x\rightarrow +\infty} g(x)\in \R;\ \Tilde{g}(x):=g(x)-b\ -$ монотонна и $\lim\limits_{x\rightarrow +\infty}\Tilde{g}(x)=0$.

    $F(y):=\int\limits_a^y f(x)dx$ и $\lim\limits_{y\rightarrow +\infty}F(y)=\int\limits_a^{+\infty}f(x)dx\in \R\Rightarrow F(y)$ ограничена при больших $y\Rightarrow F\ -$ ограниченная функция.

    Тогда $f$ и $\Tilde{g}$ удовлетворяют признаку Дирихле $\Rightarrow\int\limits_a^{+\infty} f(x)\Tilde{g}(x)dx$.

    $\int\limits_a^{+\infty} f(x)g(x)dx=\underbrace{\int\limits_a^{+\infty} f(x)\Tilde{g}(x)dx}_{\text{доказали, что сходится}}+b\underbrace{\cdot \int\limits_a^{+\infty} f(x)dx}_{\text{сходится по условию}}\Rightarrow \int\limits_a^{+\infty} f(x)g(x)dx$ сходится.
\end{proof}

\begin{corollary}
    $f,g\in C[a,+\infty),\ f$ периодична с периодом $T$, $g$ монотонна, $g(x)\underset{x\rightarrow +\infty}{\rightarrow}0$ и $\int\limts_a^{+\infty} g(x)dx$ расходится. Тогда $\int\limits_a^{+\infty}\ -$ сходится $\Leftrightarrow \int\limits_a^{a+T} f(x)dx=0$.
\end{corollary}

\begin{proof}
    $\Leftarrow:\ F(y):=\int\limits_a^y f(x)dx\ -$ периодична с периодом $T$.
    $F(y+T)=F(y)+\underbrace{\int\limits_y^{y+T}f(x)dx}_{=0}\Rightarrow$ все значения $F$ принимает на $[a,a+T]$, а там она ограничена по th В. $\Rightarrow$ можно применить принцип Дирихле.

    $\Rightarrow:$ от противного.

    Пусть $b:=\int\limitsa^{a+T}f(x)dx\neq 0$. Расссмотрим $\Tilde{f}(x):=f(x)-\frac{b}{T}\Rightarrow \int\limits_a^{a+T}\Tilde{f}(x)dx=0\Rightarrow\int\limits_a^{+\infty}\Tilde{f}(x)g(x)dx$ сходится $\Rightarrow\underbrace{\int\limits_a^{+\infty}f(x)g(x)dx}_{\text{сходится}}-\underbrace{\int\limits_a^{+\infty}\Tilde{f}(x)g(x)dx}_{\text{сходится}}=\frac{b}{T}\int\limits_a^{+\infty} g(x)dx\ -$ сходится. Противоречие.
\end{proof}

\begin{example}
    $\int\limits_p^{+\infty}\frac{\sin x}{x^p}dx$

    \begin{enumerate}
        \item Если $p>1:\ |\frac{\sin x}{x^p}|\leq \frac{1}{x^p},\ \int\limits_1^{+\infty}\frac{1}{x^p}dx$ сходится при $p>1\Rightarrow$ абсолютно сходящийся.
        \item Если $0<p\leq 1:\ \int\limits_1^{+\infty}\ -$ расходится $\frac{1}{x_p}\searrow$ при $x\rightarrow +\infty$

        $\int\limits_0^{2\pi}\sin xdx=0\overset{\Rightarrow}{\text{по следствию}}\int\limits_0^{\infty}\frac{\sin x}{x^p}$ сходится.

        $\int\limits_0^{2\pi}|\sin x|dx>0\Rightarrow\int\limits_0^{\infty}\frac{\sin x}{x^p}$ расходится.

        Т.е. сходится, но не абсолютно.
        \item Если $p\leq 0:\ a_n=\frac{\pi}{6}+2\pi n,\ b_n=\frac{5\pi}{6}+2\pi n\Rightarrow \sin x\geq \frac{1}{2}$ при $x\in[a_n, b_n]$

        $\int\limits_{a_n}^{b_n}\frac{\sin x}{x^p}dx\geq \frac{1}{2}\int\limits_{a_n}^{b_n}\frac{1}{x^p}dx\geq \frac{1}{2}(b_n-a_n)=frac{1}{2}\cdot \frac{2\pi}{3}=\frac{\pi}{3}$ т.е. сколь угодно далеко есть отрезок с $\int=\frac{\pi}{3}$.

        Противоречие с критерием Коши: $\int\limits_1^{+\infty}f(x)dx$ сходится $\Leftrightarrow \forall \underbrace{\varepsilon}_{=\frac{\pi}{3}} >0\exists B\ \forall \underbrace{a}_{=a_n},\underbrace{b}_{=b_n}>B\ |\int\limits_a^b f(x)dx|<\varepsilon$
    \end{enumerate}
\end{example}