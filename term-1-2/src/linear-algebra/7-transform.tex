% $\left(\begin{array}{c|c}A & B\\\hline C & D\end{array}\right)$
\section{Элементарные преобразования}

\begin{definition}
    $\mathcal{A}:K^n\rightarrow K^n\ -$ линейное отображение 
$(\mathcal{A}(X)=AX)$; $\mathcal{A}\ -$ \textbf{элементарное 
преобразование}, если $\exists i_0:\ \mathcal{A}(X)_i=x_i\ \forall i\neq 
i_0,\ X=\begin{pmatrix}
    x_1 \\ ... \\ x_n
    \end{pmatrix}$ и $\mathcal{A}\ -$ обратимо (меняется только одна 
координата).
\end{definition}

\begin{example}
    $\mathcal{A}\begin{pmatrix}
        x_1 \\ x_2 \\ x_3
    \end{pmatrix}=\begin{pmatrix}
        x_1 \\ \sqrt{2}x_2 + x_3 - x_1 \\ x_3
    \end{pmatrix}\ -$ элементарно.
\end{example}

\begin{statement}
    \textbf{Виды элементарных преобразований:}

    \begin{enumerate}
        \item \textbf{Трансвекция}: $t_{i,j}(a)\begin{pmatrix}
    x_1 \\ ... \\ x_n
    \end{pmatrix}=\begin{pmatrix}
    x_1 \\ ... \\ x_i+a\cdot x_j \\ ... \\ x_n
    \end{pmatrix}$

    $t_{i,j}(a)\ -$ обратимо.

    $t_{i,j}(a)^{-1}=t_{i,j}(-a)$

    Матрица: $E+a\cdot e_{i,j}$
        \item \textbf{Дилатация:} $m_i(a),\ \a\in K\setminus \{0, 1\}$

        $m_i(a)\begin{pmatrix}
    x_1 \\ ... \\ x_n
    \end{pmatrix}=\begin{pmatrix}
    x_1 \\ ... \\ a\cdot x_i \\ ... \\ x_n
    \end{pmatrix}$

    $m_{i}(a)\ -$ обратимо.

    $m_{i}(a)^{-1}=m_{i}(\frac{1}{a})$

    Матрица: $E+(a-1)\cdot e_{i,i}$

    \item Часто рассматривается третий тип $-$ \textbf{транспозиция}: 
$s_{i,j}=\begin{pmatrix}
    x_1 \\ ... \\ x_i \\ ... \\ x_j \\ ...\\ x_n
    \end{pmatrix}\rightarrow \begin{pmatrix}
    x_1 \\ ... \\ x_j \\ ... \\ x_i \\ ...\\ x_n
    \end{pmatrix}$

    Не нужна: $\begin{pmatrix}
        x \\ y
    \end{pmatrix}\rightarrow \begin{pmatrix}
        x + y \\ y
    \end{pmatrix}\rightarrow \begin{pmatrix}
        x + y \\ -x
    \end{pmatrix}\rightarrow \begin{pmatrix}
        y \\ -x
    \end{pmatrix}\rightarrow \begin{pmatrix}
        y \\ x
    \end{pmatrix} $

    $s_{1,2}=m_2(-1)\cdot t_{1,2}(1)\cdot t_{2,1}(-1)\cdot t_{1,2}(1)$ (то 
есть транспозиция $-$ это композиция трансвекций и дилатаций)
    \end{enumerate}
\end{statement}

\begin{statement}
    $A\ -$ матрица:
    \begin{enumerate}
        \item $t_{i,j}(a)\cdot A$ получается из $A$ прибавлением к $i-$ой 
строке $j-$ой строки, умноженной на $a$.
        \item $m_{i}(a)\cdot A$ получается из $A$ умножением $i-$ой строки 
на $a$.
        \item $A\cdot t_{i,j}(a)$ получается из $A$ прибавлением к $j-$ому 
столбцу $i-$ого столбца, умноженного на $a$.
        \item $A\cdot m_{i}(a)$ получается из $A$ умножением $i-$ого 
столбца на $a$.
    \end{enumerate}
\end{statement}

\begin{theorem}
    \begin{enumerate}
        \item[]
        \item[1)] $A\in M_{n,m}\Rightarrow \exists e_1,...,e_k\ -$ 
трансвекции/дилатации.

        $e_1...e_kA=\begin{pmatrix}
            a_{11} & * & * & * \\ 0 & a_{22} & * & * \\ ...  & ... & ... & 
... \\ 0 & ... & .. & ...
        \end{pmatrix}$ (треугольная матрица)
        \item[2)] $m=n$ и $A$ обратима, то $\exists 
e_1,...,e_k:e_1...e_kA=E$ 
        \item[2')] $m=n$ и $A$ обратима, то $A\ -$ произведение 
трансвекций/дислатаций.
        \item[3)] $A\ -$ произведение $\Rightarrow \exists e_1,...,e_k$ и 
$d_1,...,d_l\ -$ трансвекции/дилатации 
$e_1...e_kAd_1...d_k=\left(\begin{array}{c|c}E_r & 0\\\hline 0 & 
0\end{array}\right)$.
    \end{enumerate}
\end{theorem}

\begin{proof}

\begin{enumerate}
    \item[1)] Индукция по $n$

    Переход $n\rightarrow n + 1$: рассмотрим первый столбец 
$\begin{pmatrix}
        a_{11} \\ ... \\ a_{m1}
    \end{pmatrix}$

    \begin{enumerate}
        \item[I.] Все $a_{i1}=0;\ A=\begin{pmatrix}
            0 \\ ... \\ 0
        \end{pmatrix}\underbrace{\Tilde{A}}_{n}$

        Применим к $\Tilde{A}$ индукционное предположение: $\exists$ 
элементарные матрицы (порядка $m$) $e_1,...,e_k\Tilde{A}=\begin{pmatrix}
            a_{11} & ... & * \\ ... & a_{22} & ... \\ 0 & ... & ...
        \end{pmatrix}$

        Тогда $e_1...e_kA=\begin{pmatrix}
            0 & a_{11} & ... &  * \\ 0 & 0 & a_{22} \\ ... \\\ 0
        \end{pmatrix}$
        \item[II.] $a_{11}\neq 0:$ применим 
$t_{i,1}(-\frac{a_{i1}}{a_{11}})$ для каждого $i=2..m$.

        $e_2...e_mA=\left(\begin{array}{c|c}a & b\\ \hline 0  \\ ... & 
\Tilde{A} \\ 0 \end{array}\right)$

        Применим индукционное предположение к $\Tilde{A}:\ \exists 
\Tilde{e_{m+1}},...,\Tilde{e_s}:$ 
        
        $ \Tilde{e_{m+1}},...,\Tilde{e_s}\Tilde{A}=\begin{pmatrix}
            \Tilde{a_{11}} & .. & * \\ 0 & \Tilde{a_{22}} & ... \\ 0
        \end{pmatrix}$

        Заменим $\Tilde{e_k}$ на $e_k:\left(\begin{array}{c|c}1 & 0 \\ 
\hline 0  \\ ... & \Tilde{e_k} \\ 0 \end{array}\right)$

        Тогда $e_{m+1}...e_sA=\left(\begin{array}{c|c c}a & b\\ \hline 0 & 
\Tilde{a_{11}} \\ ... & 0 & \Tilde{a_{22}} \\ 0 \end{array}\right)$
        \item[III.] $A\ -$ квадратная обратимая матрица. По первому 
пункту: $\exists$ элементарные преобразования: $a_1...a_kA=\begin{pmatrix}
            a_{11} & ... & * \\ 0 & a_{22} \\ ... \\ 0 & ... & a_{nn}
        \end{pmatrix}$

        \underline{Утверждение:} $a_{ii}\neq 0$

        \underline{Доказательство утверждения:} пусть $\exists 
a_{ii}=0\Rightarrow $ 
        
        первые $i$ столбцов матрицы $\in \bigg\langle\begin{pmatrix}
            1 \\ 0 \\ ... \\ 0
        \end{pmatrix}, ..., \begin{pmatrix}
            0 \\ ... \\ 1_{(i-1\text{ позиция})} \\... \\ 0
        \end{pmatrix}\bigg\rangle \Rightarrow$ они ЛЗ (но у обратных 
матриц столбцы линейно независимы)

        $a_1...a_kA$ обратима, так как ранг не меняется при домножении на 
обратимые матрицы.

        Теперь применим 
$\prod\limits_{i=1}^{n-1}t_{in}(-\frac{a_{in}}{a_{nn}})$:

        $\begin{pmatrix}
            a_{11} & * & a_{1n} \\ & & a_{2_n} \\ \\ & & a_{nn}
        \end{pmatrix}\leadsto \begin{pmatrix}
            a_{11} & * & 0 \\ & & 0 \\ \\ 0 & & a_{nn}
        \end{pmatrix}$

        Потом применим 
$\prod\limits_{i=1}^{n-2}t_{i,n-1}(-\frac{a_{i,n-1}}{a_{n-1,n-1}})$ и так 
далее...

        В итоге получим: $\begin{pmatrix}
            a_{11} & ... & 0 \\ ... \\ 0 & ... & a_{nn}
        \end{pmatrix}$.

        Применим $\prod m_i(\frac{1}{a_{ii}})$; получим $\begin{pmatrix}
            1 & ... & 0 \\ ... \\ 0 & ... & 1
        \end{pmatrix}=E$.
        
    \end{enumerate}
    
    \item[2')] $\exists e_1,...,e_k;\ e_1...e_kA=E$

    $A=e^{-1}_k...e_!^{-1}$ (отметим, что $e_1...e_k=A^{-1}$)

    $e_i\ -$ элементарные преобразования
    \item[3)] $A\ -$ произвольная матрица, $\exists$ обратимые матрицы 
$C,D:\ CAD=\left(\begin{array}{c|c}E_r & 0\\\hline 0 & 
0\end{array}\right)$ 

    $C=e_1....e_k$ по 2' $\And D=d_1....d_l\Rightarrow 
e_1....e_kAd_1....d_l=\left(\begin{array}{c|c}E_r & 0\\\hline 0 & 
0\end{array}\right)$
\end{enumerate}

\end{proof}

\begin{corollary}
    Из теоремы следует алгоритм нахождения обратной матрицы:

    $e_k...e_1A=E\Rightarrow e_k...e_1=A^{-1}$

    Создаем блочную матрицу: $(A \mid E)$; применяем одновременно 
элементарные преобразования к $A$ и к $E$ (приводя $A$ к $E$)

    $e_1...e_k(A\mid E)=(e_1...e_kA\mid e_1...e_kE)\Rightarrow (A \mid 
E)\xrightarrow[]{\text{элементарные преобразования}}(E \mid A^{-1})$
\end{corollary}
