\section{Базис и размерность}
\subsection{Линейная комбинация}

\begin{definition}
    $v_1,...,v_n\in V$ — векторное пространство над $K;\ a_1,...,a_n\in 
K$. $a_1v_1+...+a_nv_n$ $—$ \textbf{\textit{линейная комбинация}} 
$v_1,..,v_n$ с коэффициентами $a_1,...,a_n$.
\end{definition}

\begin{definition}
    $v_1,...,v_n\in V$ — множество линейных комбинаций, замкнутых 
относительно $+,\cdot\Rightarrow$  является векторным пространством. Оно 
называется \textbf{\textit{линейной оболочкой}} $v_1,...,v_n - \langle 
v_1,...,v_n \rangle$. \\
    $\sum a_iv_i+\sum b_iv_i=\sum (a_i+b_i)v_i;\ k\cdot \sum a_iv_i=\sum 
(ka_i)\cdot v_i$
\end{definition}

\begin{remark}
    Все тоже для бесконечных систем $\{v_i\}_{i\in I}$. 
\textbf{\textit{Линейная комбинация}} — это $a_1v_1+...+a_nv_n$ или $\sum 
a_iv_i,$ где почти все $a_i=0$ (все, кроме конечного числа)
\end{remark}

\begin{example}
    $v, u$ — неколлинеарные векторы. 
    
    $\langle v \rangle =\{kv \}$ — прямая, содержащая $v$.
    
    $\langle v,u \rangle$  — плоскость, натянутая на $u$ и $v$.
\end{example}

\begin{definition}
    $\{v_i\}$ $-$ \textbf{\textit{линейно независимое множество 
векторов}}, если выполнено одно из двух равносильных условий:
    \begin{enumerate}
        \item $\forall i\ v_i\neq \sum\limits_{j\neq i}a_jv_j$
        \item $\sum\limits_{j}a_jv_j=0\Rightarrow \forall i\ a_i=0$, то 
есть никакая линейная комбинация $v_i$ не равна 0.
    \end{enumerate}
\end{definition}

\begin{example}
    $\{\overline{0}\}$ — линейно независимое 
    \begin{enumerate}
        \item $\overline{0} =\overline{0}=\sum\limits_{\varnothing}$
        \item $1\cdot \overline{0}=\overline{0}$
    \end{enumerate}
\end{example}
\begin{proof}
\begin{enumerate}
    \item[]
    \item[\circ] $2\Rightarrow 1$ \\
    Пусть $v_1=\sum\limits_{i\neq 1}a_iv_i\Rightarrow (-1)\cdot 
v_1+\sum\limits_{i\neq 1}a_iv_i=0$, но не все коэффициенты = 0.
    \item[\circ] $1\Rightarrow 2$ \\
    $\sum a_iv_i=0$ и $\exist i$ (НУО $i=1$): $a_i\neq 0;$ тогда 
$v_1=-\frac{a_2}{a_1}-...$
\end{enumerate}
\end{proof}

\begin{remark}
    Здесь важно, что $K$ — поле $(\frac{a_i}{a_1})$.
\end{remark}

\begin{definition}
    $R$ — ассоциативное кольцо; тогда тройка $(V,+,\cdot)$ т.ч. выполнены 
8 аксиом, называется $R$\textbf{\textit{-модулем}}.
\end{definition}

\begin{definition}
    $V$ — векторное пространство над $K;\ \{v_i\}$ — 
\textbf{\textit{порожденная система}}, если $\lang\{v_i\} \rang =V$ (т.е. 
$\forall v\in V$ — линейная комбинация).
\end{definition}

\begin{remark}
    Пусть $M\subset V=\{v_i\}$ — система векторов.
    \begin{enumerate}
        \item $M$  — линейно независимое, $N\subset M\Rightarrow N$ — 
линейно независимое.
        \item $M$  — порожденная система, $N\supset M\Rightarrow N$ — 
порожденная система.
    \end{enumerate}
\end{remark}

\subsection{Базис}

\begin{definition}
    $V$ — векторное пространство, $\{v_i\}\in V;\ \{v_i\}_{i\in I}$ — 
\textbf{\textit{базис}}, если выполнены 4 равносильных условия:
    \begin{enumerate}
        \item $\{v_i\}$ — лин. нез. и пор.
        \item $\{v_i\}$ — макс. лин. нез., т.е. $\forall v\in V\ 
\{v_i\}\cup\{v\}$ — лин. зав.
        \item $\{v_i\}$ — мин. пор., т.е. $\forall i\in I\ \{v_j\}_{j\in 
I}\setminus\{v_i\}$ — не пор.
        \item $\forall v\in V$ представляется единственным образом как 
линейная комбинация $\{v_i\}$ 
    \end{enumerate}
\end{definition}

\begin{proof} Докажем равносильность:
    \begin{enumerate}
        \item[\circ] $1\Rightarrow 2$ 
        
        $\{v_i\}\cup\{v\}\ \ \ \ \lang\{v_i\}\rang =V\Rightarrow v$ — л.к. 
$\{v_i\}\Rightarrow \{v_i\}\cup\{v\}$ лин.зав.
        \item[\circ] $2\Rightarrow 1$ 
        
        $v\in V\ \{v_i\}\cup\{v\}$ — л.з. по условию: $\exists 
a_{i_1},...,a_{i_k},a:\ a_{i_1}v_{i_1}+...+a_{i_k}v_{i_k}+av=0$ и не все 
коэффициенты $=0$.
        \begin{enumerate}
            \item[I.] Пусть $a\neq 0\Rightarrow v=-\sum 
\frac{a_{i_l}}{a}v_{i_l}$, т.е. $v\in\langle \{v_i\}\rangle$
            \item[II.] $\sum a_{i_l}v_{i_l}=0$ не все $a_{i_l}=0,$ против. 
с ЛНЗ $\{v_i\}$
        \end{enumerate}
        \item[\circ] $1\Rightarrow 4$ 
        
        $v\in V\ \ \ \{v_i\}\Rightarrow v=\sum a_iv_i$, осталось доказать 
единственность.
        
        Пусть $v=\sum a_iv_i=\sum b_iv_i\Rightarrow 0=\sum 
(a_i-b_i)v_i\Rightarrow^{\text{ЛНЗ}}a_1=b_i\ \forall i$
        \item[\circ] $4\Rightarrow 1$ 
        
        $\{v_i\}$ пор. по усл., осталось доказать ЛНЗ 
        
        $v_1=\sum\limits_{I\neq 1} a_iv_i=1\cdot v_1+0\cdot v_2+...$ 
разные разложения ???
    \end{enumerate}
TODO
\end{proof}


\begin{statement}
    $f:V\rightarrow\ ^nK\ (v\rightarrow (a_1,...,a_n))$ — изоморфизм
\end{statement}

\begin{proof}
    Kорректность и биективность $-$ по определению базиса. 
    
    Гомоморфность очев.
\end{proof}

\begin{designation}
    Базис — строка, координаты — столбец.
\end{designation}

\begin{example}
    $K[x]_2\ \ \ 1,\ x,\ x^2\ |\ x^2+1,\ x^2-x,\ x^2+x+3 \rightarrow$  
базисы. \\
    $2x^2+3\leadsto(2,0,3)\ |\ (0,1,1)\rightarrow$ представление в разных 
базисах.
    
\end{example}

\begin{definition}
    $V$ называют \textbf{\textit{конечномерным}}, если в $V \ \exists$ 
конечная порождающая система $(V=\langle v_1,...,v_n \rangle)$.
\end{definition}

\begin{lemma}
    Из любой конечной порождающей системы можно выбрать базис.
\end{lemma}

\begin{proof}
    Пусть $v_1,...,v_n$ — порожденная система; если она ЛНЗ, то вот и 
базис.
    
    Иначе $\exists v_i$ (НУО $v_n$)$:\ v_n=\sum\limits_{i=1}^{n-1}a_iv_i$
    
    Но тогда $\forall$  линейная комбинация $v_1,...,v_n$ это 
$\sum\limits_{i=1}^{n}b_iv_i=\sum\limits_{i=1}^{n-1}b_iv_i+b_n(\sum\limits_{i=1}^{n-1}a_iv_i)\in 
\langle v_1,...,v_{n-1} \rangle$ 
    
    Значит, $v_1,...,v_n$ — порождающая система. 
    
    Будем продолжать этот процесс, пока система не станет линейно 
независимой (что когда-нибудь случится, так как система была конечной).
\end{proof}

\begin{corollary}
    В любом конечном пространстве есть базис.
\end{corollary}

\begin{remark} \textbf{Лемма Цорна} \\
    На самом деле в любом пространстве есть базис.
\end{remark}

\begin{example}
    $K[x]\ \ \ \{1, x, x^2, ...\}$ — базис

    $K[[x]]\ -$  базис существует, но конструктивно его не предъявить

    $\R$ над $\mathbb{Q}\ -$ базис есть, но…
\end{example}

\begin{definition}
     $V$ — векторное пространство (конечномерное); 
\textbf{\textit{размерность}} $V\ (dim\ V)$  — это количество векторов в 
его базисе.
\end{definition}

\begin{theorem}
    В двух любых базисах $V$ поровну элементов.
\end{theorem}

 \begin{proof}
    Это следует из леммы.
 \end{proof}

 \subsection{Лемма о линейной зависимости линейных комбинаций}

\begin{lemma} \textbf{о линейной зависимости линейных комбинаций (ЛЗЛК)} 
\\
    Пусть $u_1,...,u_n\in \lang v_1,...,v_m\rang, \ m>n$. Тогда 
$u_1,...,u_m$ линейно зависима.
\end{lemma}

\begin{proof}
    \textit{Лирическое отсутупление:}
    
    В теореме: пусть $v_1,...,v_m$ — базис min размера 
    
    $\exists$ базис $u_1,...,u_{n+1}$ 
    
    Все $u_i$ — л.к. $\{v_i\}$, т.к. $\{v_i\}$ — базис $\Rightarrow 
\{u_i\}$ — л.з. ??? \\
    \textit{Само доказательство:}
    
    НУО: $n=m+1\ \ \ (u_1,...,u_{m+1}$ — л.з. $\Rightarrow u_1,...,u_{n}$ 
— л.з.$)$ \\
    Индукция по $m$:
    \begin{enumerate}
        \item База: $m=1;\ \ u_1=a_1v_1,\ u_2=a_2v_1$ 
        
        \textit{Два случая:}
        \begin{enumerate}
            \item[1)] $a_1$ или $a_2=0\Rightarrow \{u_1,u_2\}$ — л.з.
            \item[2)] $a_1,a_2\neq 0\Rightarrow \frac{a_2}{a_1}\cdot u_1,$ 
т.е. опять л.з.
        \end{enumerate}
        \item Переход: $m\rightarrow m+1$ 
        
        $u_1,...,u_{m+2}\ \ \ v_1,...,v_{m+1}$ 
        
        $u_i=\sum\limits_{j=1}^{m+1}a_{i_j}v_j$ — л.к.. Далее возможны 
случаи:
        \begin{enumerate}
            \item[1)] Пусть $a_{i_{m+1}}=0\ \forall i\Rightarrow 
u_1,...,u_{m+2}\in\langle 
v_1,...,v_{m}\rangle\Rightarrow^{\text{и.п.}}u_1,...,u_{m+1}$ — л.з.
            \item[2)] Пусть $\exist i:a_{i_{m+1}}\neq 0,$ НУО 
$a_{1_{m+1}}\neq0$ \\            $u_1=a_{1_1}v_1+...+a_{1_{m+1}}v_{m+1}$
            
            $u_2=a_{2_1}v_1+...+a_{2_{m+1}}v_{m+1}$
            
            …
            
            $u_{m+2}=a_{{m+1}_1}v_1+...+a_{{m+1}_{m+1}}v_{m+1}$ 
            
            $\forall k=2...m+2$ из $k$-ого равенства вычтем 1-ое, 
умноженное на $\frac{a_{k_{m+1}}}{a_{1_{m+1}}}$.
            
            $\tilde{u}_i=u_i-\frac{a_{k_{m+1}}}{a_{1_{m+1}}}\cdot 
u_1=(a_{i_1}-\frac{a_{1_1}\cdot a_{i_{m+1}}}{a_{1_{m+1}}})\cdot 
v_1+...+(a_{i_{m}}-\frac{a_{1_m}\cdot a_{i_{m+1}}}{a_{1_{m+1}}})\cdot 
v_m-(a_{i_{m+1}}-a_{i_{m+1}})\cdot v_{m+1}$
            
            Получили $\tilde{u}_2,...,\tilde{u}_{m+1}\in \langle v_1, 
...,v_m\rangle$. Тогда по и.п. $\tilde{u}_{2},...,\tilde{u}_{m+2}$ — л.з.
            
            То есть $\exists b_2,...,b_{m+2}$ не все равные 0:
            
            $0=\sum\limits_{i=2}^{m+2}b_i\tilde{u}_i=\sum 
b_i(u_i-...u_{1})=(-\sum\frac{a_{i_{m+1}}}{a_{1_{m+1}}})\cdot 
u_1+b_2u_2+...+b_{m+2}u_{m+2}$ — нетривиальная линейная комбинация.
        \end{enumerate}
    \end{enumerate}
\end{proof}

\begin{lemma}
    $u$ — конечномерное векторное пространство; $u_1,...,u_k\in U$ — ЛНЗ 
система $\Rightarrow \exists u_{k+1},...,u_n-$ базис $U$
    
    (любую ЛНЗ систему можно дополнить до базиса)
\end{lemma}

\begin{proof}
    $u_1,...,u_k\ -$ ЛНЗ $\Rightarrow \left[\begin{gathered} \text{макс. 
ЛНЗ}\Rightarrow \text{это базис}; \\ \exists u_{l+1}:u_1,...,u_{l+1}\ 
-\text{ЛНЗ} \end{gathered}\right.$
    
    Будем добавлять к $u_1,...,u_k$ по вектору. $l$ не может стать больше 
$n$* $\Rightarrow$  в какой-то момент получим базис.
    
    *$u_1,...,u_n,u_{n+1}\ -$ ЛЗ по ЛЗНК $($рассмотрим $v_1,...,v_n\ -$ 
базис $U;u_1,...,u_{n+1}\in\langle v_1,...v_n\rangle\Rightarrow 
u_1,...,u_{n+1}\ -$  ЛЗ$)$
\end{proof}

\begin{corollary}
    $U,V-$ конечномерное векторное пространство над $K$ и $U\leq 
V\Rightarrow \dim U\leq \dim V$ и если $\dim U=\dim V\Rightarrow U=V$ 
\end{corollary}

\begin{proof}
    $u_1,...,u_k-$ базис $U;$ по лемме можем дополнить до $u_1,...,u_n\ -$ 
базис $V\Rightarrow k\leq n$ 
    
    Если $k=n,$ то дополняем 0 векторов $\Rightarrow$  базис $U$ = базис 
$V\Rightarrow$ оба пространства — линейные комбинации одних и тех же 
векторов
\end{proof}

\subsection{Алгебраические числа}

\begin{definition}
    $\mathbb{Q};\ \alpha\ -$ алгебраическое число, если $\exists f\in 
\mathbb{Q}[x]:f(\alpha)=0$
\end{definition}

\begin{example}
    $\sqrt[7]{3}-$ алг., т.к. $\exists f(x)=x^7-3$ 
    
    $\pi, e-$не алг. (не знаем)
\end{example}

\begin{theorem}
    $\alpha\ -$  алгебраическое, $P\in\Z[x]\Rightarrow 
P(\alpha)-$алгебраическое
\end{theorem}

\begin{proof}
    Рассмотрим $V_{\alpha}-$ в.п. над $\mathbb{Q}:\{P(\alpha)\ |\ 
P\in\mathbb{Q}[x]\}-$ замкнуто относительно $+$ и $\cdot$    на 
рациональные числа (т.е. это в.п.)
    \begin{statement}
        Это пространство конечномерное.
    \end{statement}
    \begin{proof}
        $\exists 
a_0,...,a_{n-1}\in\mathbb{Q}:\alpha^n+a_{n-1}\alpha^{n-1}+...+a_1\alpha+a_0=0$
        
        
$\alpha^n=-\sum\limits_{i=0}^{n-1}a_i\alpha^{i}\in\langle1,\alpha,...,\alpha^{n-1}\rangle 
\Rightarrow 
\langle1,\alpha,...,\alpha^n\rangle=\langle1,\alpha,...,\alpha^{n-1}\rangle$ 
        
        
$\alpha^{n+1}=-\sum\limits_{i=0}^{n-1}a_i\alpha^{i+1}\in\langle1,\alpha,...,\alpha^{n}\rangle\in\langle1,\alpha,...,\alpha^{n-1}\rangle 
\Rightarrow 
\langle1,\alpha,...,\alpha^{n+1}\rangle=\langle1,\alpha,...,\alpha^{n-1}\rangle$ 
        
        Продолжим так делать и получаем, что $\forall N\ 
\alpha^{N}\in\langle 1,\alpha,..., \alpha^{n-1}\rangle$, т.е. $\text{dim 
}V_{\alpha}\leq n$        
    \end{proof}
    $\exists P\in\Z[x]:1,P(\alpha),...,(P(\alpha))^n\in V_{\alpha}$, их 
$n+1,\dim V_{\alpha}\Rightarrow1,...,(P(\alpha))^n-$ ЛЗ, т.е. $\exists 
q_0,...,q_n:q_0\cdot1+q_1\cdot  P(\alpha)+...+q_n\cdot (P(\alpha))^n=0$, 
т.е. $P(\alpha)-$ корень многочлена $q_nx^n+...+q_1x+q_0,$ т.е. 
$P(\alpha)$ — алг.
\end{proof}

\begin{remark}
    Аналогично доказывается, что $\alpha,\ \beta-$ алгебраические 
$\Rightarrow \alpha+\beta,\ \alpha\beta-$ алгебраические.
\end{remark}

\begin{theorem}
    $V\ -$ конечномерное векторное пространство над $K;$ тогда 
$\exists!n:V\cong K^n$
\end{theorem}

\begin{proof} 
    \underline{Единственность:}
    
    ясно; $n=\dim K^n=\dim V$
    
    \underline{Существование:} 
    
    Пусть $u_1,u_2,...,u_n\ -$ базис $V$.
    
    Рассмотрим отображение $i:K^n\rightarrow V$:
    $\begin{pmatrix}
    a_1\\...\\ a_n
    \end{pmatrix}\rightarrow a_1u_1+...+a_nu_n$
    
    Это биекция (т.к. $u_1,...,u_n,$ а координаты единственны) и 
гомоморфизм (по очеву).
\end{proof}

\begin{designation}
    $u\in V,\ \mathcal{U}=\begin{pmatrix}
    a_1\\...\\ a_n
    \end{pmatrix}=[u]_{\{u_i\}}$ — столбец координат в базисе $\{u_i\}$. 
    
    $[u]$ зависит от $\{u_i\}.$
\end{designation}

\begin{example}
    Фибоначчиева последовательность: \\
    $(1,0,1,1,2,3,...)=v_1$ 
    
    $(0,1,1,2,3,5,...)=v_2$ 
    
    $(a,b,a+b,2b+a,...)=av_1+bv_2\leadsto \begin{pmatrix} a \\ b 
\end{pmatrix}$ \\
    Классическое: $(1,1,2,3,...)\leadsto \begin{pmatrix} 1 \\ 1    
\end{pmatrix}$ 
\end{example}

\begin{example}
    Пример хорошего базиса: 
    
    $(1,\varphi,\varphi^2,...)=u_1$
    
    $(1,-\frac{1}{\varphi},-(\frac{1}{\varphi})^2,...)=u_2$ 
    
    Найти явную формулу для фибоначчиевой последовательности 
$\Leftrightarrow$  найти координаты в базисе $\{u_1,u_2\}:$ 
    
    $(c_1,c_2,...,c_n)=ku_1+lu_2\Rightarrow 
c_n]k\varphi^n+l\cdot(-\frac{1}{\varphi})^n$    
\end{example}
