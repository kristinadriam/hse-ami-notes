\section{Линейные отображения}

\subsection{Определение}

\begin{definition}
    $U$ и $V\ -$ векторное пространство над $K$; $\mathcal{A}:U\rightarrow 
V$  \textbf{\textit{линейно}} (гомоморфизм), если $\mathcal{A}(u+\alpha 
v)=\mathcal{A}(u)+\alpha \mathcal{A}(v)\ \forall u,v\in U,\ \forall \alpha 
\in K$.
\end{definition}

\begin{example}
    $\mathcal{A}(x)=x\ -$ линейно, $\mathcal{A}(x)=0\ -$ линейно.
\end{example}

\begin{example}
    $\mathcal{A}:K^n\rightarrow K,\ \dim_K K=1,\ (\langle1\rangle=K)$

    $\mathcal{A}\cdot 
\begin{pmatrix}x_1\\...\\x_n\end{pmatrix}=a_1x_1+...+a_nx_n;\ a_1,...,a_n\ 
-$ фиксированы.

    Более общо: $A\in M_{m,n}:A(x)=\mathcal{A}\cdot x\ -$ линейное 
отображение $\mathcal{A}:K^n\rightarrow K$.
\end{example}

\begin{remark}
    На самом деле, все линейные отображения таковы (т.е. представляют из 
себя домножение на матрицу).
\end{remark}

\begin{theorem}
    $u_1,...,u_n\ -$ базис $U;\ v_1,...,v_n\in V;$  тогда $\exists !\text{ 
линейное отображение }\mathcal{A}:U\rightarrow V \text{ т.ч. } 
\mathcal{A}(u_i)=v_i$.
\end{theorem} 

\begin{proof}
    \underline{Единственность:}    

    Пусть $\mathcal{A}(u_i)=v_i=\mathcal{B}(u_i)$.

    Пусть $\forall u\in U: u=\sum a_iu_i\Rightarrow 
\mathcal{A}(u)=\mathcal{A}(\sum a_iu_i)=\sum a_i\mathcal{A}(u_i)=\sum 
a_i\mathcal{B}(u_i)=B(\sum a_iu_i)=\mathcal{B}(u)$

    \underline{Существование:}

    Для каждого $u=\sum a_iu_i$ положим $\mathcal{A}(u)=\sum a_iv_i$:

    $\mathcal{A}\ -$ линейно: $\mathcal{A}(\alpha\cdot (\sum a_iu_i)+\sum 
b_iu_i)=\mathcal{A}(\sum(\alpha\cdot a_i+b_i)u_i)=\alpha\cdot \sum 
a_iv_i+\sum b_iv_i=\alpha \cdot \mathcal{A}(\sum a_iu_i)+(\sum b_iu_i)$
\end{proof}

\subsection{Ядро и образ линейного отображения}

\begin{definition}
    $A:U\rightarrow V-$ линейное отображение.

    $\text{Ker }\mathcal{A}=\{u\in U\ |\ \mathcal{A}(u)=0\}\ -$ 
\textit{\textbf{ядро}} $\mathcal{A}$.

    $\text{Im }\mathcal{A}=\{v\in V\ |\ \exists u\in U:\mathcal{A}(u)=v\}\ 
-$ \textbf{\textit{образ}} $\mathcal{A}$.
\end{definition}

\begin{lemma}
    $\text{Ker } \mathcal{A}\ -$ подпространство в  $U;\text{ Im 
}\mathcal{A}-$ подпространство в $V$.
\end{lemma}

\begin{proof}
    Проверка замкнутости:

    $1)\ u,v\in \text{Ker }\mathcal{A}\Rightarrow \mathcal{A}(u)=0\And 
\mathcal{A}(v)=0\Rightarrow \mathcal{A}(u+kv)=\mathcal{A}(u)+k\cdot 
\mathcal{A}(v)=0\Rightarrow u+v,\ ku\in \text{Ker }\mathcal{A}$

    $2)\ u=\mathcal{A}(x),\ v=\mathcal{A}(y)$

    $u+v=\mathcal{A}(x)+\mathcal{A}(y)=\mathcal{A}(x+y)\Rightarrow x+y\in 
\text{Im }\mathcal{A}$

    $ku=k\cdot A(x)=\mathcal{A}(kx)\Rightarrow kx\in \text{Im 
}\mathcal{A}$
\end{proof}

\begin{example}
    $V=U=\R^2$
    \begin{enumerate}
        \item [$\circ$] $\mathcal{A}(x)=0$ 

        $\text{Ker }\mathcal{A}=\{0\},\ \text{Im }\mathcal{A}=V$

        \item[$\circ$]$\mathcal{A}(x)=0$
    
        $\text{Ker }\mathcal{A}=\{V\},\ \text{Im }\mathcal{A}=0$
    
        \item[$\circ$]$u\ -$ вектор
    
        $u\leadsto u_0-$ проекция на ОХ
    
        $\text{Ker 
}\mathcal{A}=\bigg\langle\begin{pmatrix}0\\1\end{pmatrix}\bigg\rangle,\ 
\text{Im 
}\mathcal{A}=\bigg\langle\begin{pmatrix}1\\0\end{pmatrix}\bigg\rangle$
    \end{enumerate}
\end{example}

\begin{theorem}
    $\mathcal{A}:U\rightarrow V\ -$ линейное отображение; тогда $\dim 
\text{Ker }\mathcal{A}+\dim \text{Im }\mathcal{A}=\dim U$.
\end{theorem}

\begin{proof}
    $\dim \text{Ker }\mathcal{A}=m;\ u_1,..,u_n\ -$ базис $U; \dim U=n$

    Применим $\mathcal{A}:\mathcal{A}(u_1)=...=\mathcal{A}(u_m)=0$

    Сначала докажем следующее утверждение:
    
    \begin{statement}
        $\mathcal{A}(u_{m+1}),...,\mathcal{A}(u_n)\ -$ базис $\text{Im 
}\mathcal{A}$.
    \end{statement}
    \begin{proof}
        $1)$ Докажем, что $\mathcal{A}(u_{m+1}),...,\mathcal{A}(u_n)\ -$ 
ЛНЗ

        Пусть 
$\alpha_{m+1}\mathcal{A}(u_{m+1})+...+\alpha_n\mathcal{A}(u_n)=0$

        $\mathcal{A}(\sum\alpha_{m+i}u_{m+i})=0\Rightarrow \sum 
\alpha_{m+i}u_{m+i}\in \text{Ker }\mathcal{A}\Rightarrow \sum 
\alpha_{m+i}u_{m+i}=\sum\limits_{i=1}^m\alpha_{i}u_{i}\ (\text{т.к. 
}u_1,...,u_m\ -\text{базис Ker }\mathcal{A})\  ??$

        $\sum \alpha_{m+i}u_{m+i}-\sum\alpha_{i}u_{i}=0;$ т.к. $\{u_i\}\ 
-$ ЛЗ, то все $\alpha_i=0$.
    \end{proof}
    TODO
\end{proof}

\subsection{Формула Грассмана}
\textbf{Применение:} Формула Грассмана. В множествах есть базовые операции 
$\cup,\ \cap$. В пространствах $-$ $\cap,\ +$.

\begin{definition}
     $V_1+V_2=\{v_1+v_2\ |\ v_1\in V_1,v_2\in V_2\}\ -$ 
\textbf{\textit{сумма подпространств}} (сумма Минковского).
\end{definition}

\begin{theorem}
    $V\ -$ векторное пространство над $K$; $V_1,V_2$ $-$ подпространства 
$(V_1,V_2\leq V)$. 
    \begin{enumerate}
        \item $V_1\cap V_2\ -$ подпространство $V$ (очев).
        \item $V_1+V_2\ -$ тоже подпространство.
    \end{enumerate}    
\end{theorem}
\begin{proof}
     $v_1\in V_1+V_2\Rightarrow v=v_1+v_2\And v'_1\in V_1+V_2\Rightarrow 
v'=v'_1+v'_2\Rightarrow v+v'=(v_1+v'_1)+(v_2+v'_2)\in V\ (v_1+v'_1\in 
V_1,\ v_2+v'_2\in V_2\ $т.к. $V_1,V_2\leq V)$

    $k\cdot v=k\cdot (v_1+v_2)=k\cdot v_1+k\cdot v_2\in V,\ kv_1\in V_1,\ 
kv_2\in V_2$
\end{proof}

\begin{remark}
    $v_1,...,v_n\ -$ базис $V_1$, $v'_1,...,v'_m\ -$ базис $V_2\Rightarrow 
\{v_1,...,v_n,v'_1,...,v'_m\}\ -$ порождающая система $V_1+V_2$.
\end{remark}

\begin{remark}
    Многие формулы про $+$ и $\cap \-$ аналог формул про $\cap$ и $\cup$. 
Но $(V_1+V_2)\cap V_3\neq(V_1\cap V_3)+(V_2\capV_3)$.
\end{remark}

\begin{definition}
    $V_1,V_2$ $-$ векторное пространство над $K$, 
\textbf{\textit{внешняя/прямая сумма}} $V_1$ и $V_2$ $-$ это $V_1\oplus 
V_2=\{(v_1,v_2)\ |\ v_1\in V_1,v_2\in V_2\}$ с операциями:
    \begin{enumerate}
        \item $(v_1,v_2)+(v'_1,v'_2)=(v_1+v'_1,v_2+v'_2)$
        \item $k(v_1,v_2)=(kv_1,kv_2)$
    \end{enumerate}
\end{definition}

\begin{statement}
    $\dim (V_1\oplus V_2)=\dim V_1 + \dim V_2$
\end{statement}

\begin{proof}
    Это векторное пространство (очев).

    $v_1,...,v_k\ -$ базис $V_1$, $v'_1,...,v'_m\ -$ базис $V_2$; тогда 
$\{(v_i,0)\}\cup \{(0,v'_i)\}\ -$ базис $V_1+V_2$.

    И вправду: $\forall (v,v')=(\sum a_iv_i, \sum b_iv'_i)=\sum 
a_i(v_1,0)+\sum b_i(0,v'_i)\ -$ доказали порождаемость.

    Доказательство линейной независимости: $\sum a_i(v_i,0)+\sum 
b_i(0,v'_i)=0=(\sum a_iv_i,\sum b_iv'_i)=(0,0)\Rightarrow \sum 
a_iv_i=0,\sum b_iv'_i\Rightarrow $ т.к. $v_i$ и $v'_i\ -$ базисы: все 
$a_i=0$ и все $b_i=0\Rightarrow $ доказали ЛНЗ-ть.
\end{proof}

\begin{theorem} \textbf{Формула Грассмана}

    $\dim(V_1+ V_2)=\dim V_1+\dim V_2-\dim (V_1\cap V_2)$
\end{theorem}

\begin{proof}
    Зададим $\mathcall{A}:V_1\oplus V_2\rightarrow V,\ 
(v_1,v_2)\rightarrow v_1+v_2$
    
    $\Im \mathcal{A}=V_1+V_2\ ($по определению $V_1+V_2)$

    $\Ker \mathcal{A}=\{(v,-v)\ |\ v\in V_1,\ -v\in V_2\}=\{(v,-v)\ |\ 
v\in V_1\cap V_2\}\cong V_1\cap V_2$

    Тогда по теореме о ядре и образе: $\dim 
(V_1+V_2)=\dim(\Im\mathcal{A})=\dim(V_1\oplus V_2)-\dim (V_1\cap V_2)=\dim 
V_1+\dim V_2-\dim (V_1\cap V_2)$
\end{proof}

\subsection{Множество линейных отображений}

\begin{definition}
    $U,V$ $-$ векторные пространства над $K$; определим $\Lin(U,V)$ $-$ 
\textbf{\textit{множество отображений}} $\mathcal{A}:U\rightarrow V$.

    Это векторное пространство: 
    
    $(\mathcal{A}+\mathcal{B})(u)=\mathcal{A}(u)+\mathcal{B}(u)$

    $(k\mathcal{A})(u)=k\cdot \mathcal{A}(u)$
\end{definition}

\begin{definition}
    Пусть $\mathcal{A}\in \Lin(U,V),\ u_1,...,u_m\ -$ базис $U,\ 
v_1,...,v_n\ -$ базис $V$.

    $\mathcal{A}(u_1)=a_{11}v_1+...a_{n1}v_n$

    $\mathcal{A}(u_2)=a_{12}v_1+...a_{n2}v_n$

    ...

    $\mathcal{A}(u_m)=a_{1m}v_1+...a_{nm}v_n$

    Тогда $A=(a_{ij})_{i=1..n, j=1..m}\ - $ матрица отображения 
$\mathcal{A}$ в базисе $\{v_i\}\{u_i\}$.
\end{definition}

\begin{designation}
    $A=[\mathcal{A}]_{\{u_i\}\{v_i\}}$.
\end{designation}

\begin{lemma}
    $\mathcal{A}\in \Lin (U, B),\ \{u_i\},\{v_i\}\ -$ базисы.
    
    $A=[\mathcal{A}]_{\{u_i\}\{v_i\}};\ \mathcal{A}:U\rightarrow V$ 

    $u\in U: \textbf{\mathbm{u}}=\begin{pmatrix}
        x_1 \\ ... \\ x_n
    \end{pmatrix}\ -$ координаты $u$ в $\{u_i\}$.

    Аналогично $v\in V: \textbf{\mathbm{v}}\ -$ координаты $v$ в 
$\{v_i\}$.

    Тогда $\textbf{\mathbm{v}}=A\cdot \textbf{\mathbm{u}}$.
    
\end{lemma}

\begin{proof}
    $\mathcal{A}(u)=\mathcal{A}(\sum 
x_iu_i)=\sum_{i=1}^mx_i(\sum_{j=1}^na_{ji}v_j)=\sum_{j=1}^n(\sum_{i=1}^mx_ia_{ji})v_j$ 
$-$ координаты $\textbf{\mathbm{v}}$ в базисе $\{v_i\}$ $-$ это 
$\begin{pmatrix}
        \sum_{i=1}^mx_ia_{1i} \\
        ... \\
        \sum_{i=1}^mx_ia_{mi}
    \end{pmatrix}=A\cdot \textbf{\mathbv{u}}$.
\end{proof}

\begin{corollary}
    
$U_{\{u_i\}}\xrightarrow{\mathcal{B}}V_{\{v_i\}}\xrightarrow{\mathcal{A}}W_{\{w_i\}}\ 
\mathcal{A}, \mathcal{B}\in \Lin$

    $B=[\mathcal{B}]_{\{u_i\}\{v_i\}},\ A=[\mathcal{A}]_{\{v_i\}\{w_i\}}$

    Тогда:\begin{enumerate}
        \item $A\circ B\in \Lin(U, W)$ (очев)
        \item $[A\circ B]_{\{u_i\}\{w_i\}}=A\cdot B$
    \end{enumerate}
\end{corollary}

\begin{proof}
    $u\in U,\ v\in \mathcal{B}(u),\ w\in \mathcal{A}(v)$

    $\textbf{\mathbv{u}}, \textbf{\mathbv{v}}, \textbf{\mathbv{w}}$ $-$ 
координаты $u,v,w$

    $w=(A \circ B)(u)$ и $\textbf{\mathbv{w}}=A\cdot 
\textbf{\mathbv{u}}=A\cdot B\cdot \textbf{\mathbv{v}}\Rightarrow A\cdot B$ 
$-$ матрица отображение $A\circ B$. 
\end{proof}

\begin{statement}
    $AX=BX\ \forall x\Rightarrow A=B$.
\end{statement}


\textbf{Резюме:}
\begin{enumerate}
    \item $U, V$ $-$ векторное пространство $K$ $\dim U=m,\ \dim V=m$. 
Тогда $\exists$ изоморфизм:

    $f:M_{n,m}(K)\cong \Lin (U, V)$ как векторные пространства.
    \item $n=m\ M_n(K)\cong \Lin (U, U)$ как кольца.
\end{enumerate}

\begin{remark}
    $\Lin (U, U)$ $-$ кольцо с операциями $+$ и $\circ$:

     $(\mathcal{A}+\mathcal{B})\circ \mathcal{C}=\mathcal{A}\circ 
C+\mathcal{B}\circ \mathcal{C}$ $-$ по определению $+$.

     $(\mathcal{A}+\mathcal{B})(\mathcal{C}(x))=\mathcal{A} 
(\mathcal{C}(x))+\mathcal{B}(\mathcal{C}(x))$ $-$ по определению $+$.

     $\mathcal{A}\circ(\mathcal{B}+ \mathcal{C}=\mathcal{A}\circ 
\mathcal{B}+\mathcal{A}\circ \mathcal{C}$ $-$ по определению $+$ и 
линейности $\mathcal{A}$.
\end{remark}

\begin{proof}
    \begin{enumerate}
        \item[]
        \item Фиксируем базисы $\{u_i\}$ и $\{v_i\}$ и рассмотрим $f:\Lin 
(U, V)\rightarrow M_{n,m}(K),\ \mathcal{A}\rightarrow 
[A]_{\{u_i\},\{v_i\}}$

        Очев, что это биекция (по теореме о задании линейного отображения 
на базисе). Очев, что операции сохраняются:

        $[\mathcal{A}+\mathcal{B}]=[\mathcal{A}]+[\mathcal{B}];\ 
[k\mathcal{A}]=k[\mathcal{A}]$
        \item Фиксируем базис $\{u_i\}$ и сопоставляем 
$\mathcal{A}\rightarrow [A]_{\{u_i\},\{u_i\}}$.
    \end{enumerate}
\end{proof}

\subsection{Матрица перехода}

Напоминание: $u_1,...,u_n$ $-$ старый базис, $u'_1,...,u'_n$ $-$ новый 
базис и $u_i=\sum a_{ji}u'_j$; тогда $A=(a_{ij})$ $-$ матрица перехода.

\begin{remark}
    $u_1=u_1,...,u_n=u_n\Rightarrow E=\begin{pmatrix}
        1 & 0 & ... & 0 \\
        0 & 1 & ... & ... \\
        ... & ... & 1 & 0 \\
        0 & ... & 0 & 1
    \end{pmatrix},\ E=[id]_{\{u_i\},\{v_i\}}$
\end{remark}

\subsection{Формула замены матрицы отображения при замене базиса}
$U\xrightarrow{\mathcal{A}}V,\ \{u_i\},\{v_i\}\ -$ старые базисы, 
$\{u'_i\},\{v'_i\}\ -$ новые базисы.

Знаем: $A=[\mathcal{A}]_{\{u_i\},\{v_i\}}$

Хотим: $\tilde{A}=[\mathcal{A}]_{\{u'_i\},\{v'_i\}}$

$U_{\{u'_i\}}\xrightarrow{id}U_{\{u_i\}}\xrightarrow{\mathcal{A}}V_{\{v_i\}}\xrightarrow{id}V_{\{v'_i\}}$

По следствию: 
$[\mathcal{A}]_{\{u_i\},\{v_i\}}=[id]_{\{v_i\},\{v'_i\}}\cdot 
[\mathcal{A}]_{\{u_i\},\{v_i\}}\cdot [id]_{\{u_i\},\{u'_i\}}=D\cdot A\cdot 
C^{-1}$, где $D\ -$ матрица перехода от $v_i$ к $v'_i$, а $C^{-1}\ -$ 
матрица перехода от $u_i$ к $u'_i$.

\textbf{Частный случай:} $U=V,\ \{u_i\}=\{v_i\},$ тогда 
$\tilde{A}=CAC^{-1}.$ \\



\textbf{Вопрос:} $\mathcal{A}\in \Lin (U,V),\ 
A=[\mathcal{A}]_{\{u_i\},\{v_i\}}$. Насколько простой можно сделать $A$ за 
счет замены базиса? 

\textbf{Ответ:} в теореме о $\Im$ и $\Ker$ доказали: $\exists$ базис 
$u:u_1,...,u_m,...,u_n;\ u_1,...,u_n\ -$ базис $\Ker\mathcal{A},\ 
\mathcal{A}(u_{m+1}),...,\mathcal{A}(u_n)\ -$ базис $\Im \mathcal{A}$.

Обозначим $\mathcal{A}(u_{m+1})=v_1,...,\ \mathcal{A}(u_n)=v_{n-m};\ 
v_1,...,v_{n-m}\ -$ базис $\IM \mathcal{A}$. Дополним до базиса 
$V:v_1,...,v_{n-m},...,v_l$. Тогда:

$\mathcal{A}(u_1)=0\cdot v_1+...+0\cdot v_l$

...

$\mathcal{A}(u_m)=0\cdot v_1+...+0\cdot v_l$

$\mathcal{A}(u_{m+1})=1\cdot v_1+...+0\cdot v_l$

...

$\mathcal{A}(u_n)=0\cdot v_1+...+1\cdot v_{n-m}+...+0\cdot v_l$

Получим: $\begin{pmatrix}
    0 & ... & 0 & | & 1 & 0 & ... & 0 & 0 \\
    0 & ... & 0 & | & 0 & 1 & ... & 0 & 0 \\
    \\
    0 & ... & 0 & | & 0 & 0 & ... & 1 & 0 \\
    0 & ... & 0 & | & 0 & 0 & ... & 0 & 1 \\
    \\
    0 & ... & 0 & | & 0 & 0 & ... & 0 & 0 
\end{pmatrix}$ $-$ матрица, где первые $m$ столбцов - нулевые, а верхний 
левый блок размера $(n-m)\times(n-m)$ представляет из себя нулевую матрицу 
с единицами на диагонали. Поменяем местами блоки $(u_1,...,u_m)$ и 
$(u_{m+1},...,u_n)$. Получили теорему:

\begin{theorem}
    $\forall \mathcal{A}\in \Lin (U,V)\ \exists $ базисы 
$\{u_i\},\{v_i\}:\ [A]_{\{u_i\}\{v_i\}}=\left(\begin{array}{c|c}E_s & 
0\\\hline 0 & 0\end{array}\right)$ 
\end{theorem}

\subsection{Ранг матрицы}
\begin{definition}
    Переформулировка: $\forall A\in M_{l,k}(K)\ \exists$ такие обратные 
матрицы $D$ и $C$, что $DAC=\left(\begin{array}{c|c}E_s & 0\\\hline 0 & 
0\end{array}\right)$.

    Число $s$ равно $\dim\Im \mathcal{A}$. Оно называется 
\textbf{\textit{рангом отображения}} или \textbf{\textit{рангом матрицы}}.
\end{definition}


\begin{definition}
    Матричное определение: $A\in M_{l,n}(K);$ столбцы $K^l:A=(C_1\ |\ ...\ 
|\ C_n)\ C_i\in K^l$.

    $\text{rank } A=\text{rk } A=\text{rg } A = \dim \langle 
C_1,...,C_n\rangle$
\end{definition}

\begin{remark}
    Это то же самое, что и ранг отображения, т.к. $C_i\ -$ столбец 
координат для $\mathcal{A}(u_i),$ где $u_i\ -$ $i$ый базовый вектор.
\end{remark}

\begin{example}
    $\rk \begin{pmatrix}
        1 & 2 \\
        2 & 4
    \end{pmatrix}= \rk \bigg\langle \begin{pmatrix}
        1  \\
        2
    \end{pmatrix}, \begin{pmatrix}
        2  \\
        4
    \end{pmatrix}\bigg\rangle = \rk \begin{pmatrix}
        1  \\
        2
    \end{pmatrix}, 2 \begin{pmatrix}
        1  \\
        2
    \end{pmatrix}\bigg\rangle = \rk \bigg\langle \begin{pmatrix}
        1  \\
        2
    \end{pmatrix} \bigg\rangle = 1$
\end{example}

\begin{definition}
    $U=V,\ \dim U=n, \ \rk \mathcal{A}=k;\ n-k\ -$ 
\textbf{\textit{дефект}} $\mathcal{A};\ n-k=n=\dim\Im \mathcal{A}=\dim\Ker 
\mathcal{A}$.
\end{definition} 

\subsection{Решение СЛУ}
Что такое $\dim \Ker \mathcal{A} = \dim \Ker A$?

$\{x\in K^n\ |\ AX=0\}$ $-$ множество решений однородных СЛУ с матрицей 
$A\in M_{m,n}$: 

$A\cdot\begin{pmatrix}
     x_1 \\
     ... \\
     x_n
\end{pmatrix}=\begin{pmatrix}
    0 \\
    ... \\
    0
\end{pmatrix}\ k^n\rightarrow K^m,\ n$ неизвестных, $m$ уравнений.

$\dim \Ker A\ -$ размерность пространства решений; $\dim \Ker A = n - \dim 
\Im A \geq n - m$.

Эти рассуждения подвели нас к формулированию теоремы:

\begin{theorem}
    ОСЛУ с $n$ неизвестными и $m$ уравнениями $(m<n)$ имеет пространство 
решений размерности хотя бы $n-m$; в частности, если $K$ бесконечно, то 
бесконечно много решений; если же $|K|=q\Rightarrow q^{n-m}$ решений.
\end{theorem}

Теперь рассмотрим частный случай:

\begin{theorem}
    Пусть $n=m$. Тогда: $\dim\Ker A = n - \dim \Im A, \dim \Ker 
A=0\Leftrightarrow \dim \Im A = n$, т.е. система $AX=B$ имеет решение 
$\forall B\Leftrightarrow AX=0$ имеет только тривиальное решение.

    На языке линейных отображений: $\mathcal{A}:U\rightarrow V$ линейное, 
$\dim U, \dim V<\infty$; тогда $\mathcal{A}$ сюръективно $\Leftrightarrow 
\mathcal{A}$ инъективно.
\end{theorem}

\begin{proof}
    Сюръективность $\mathcal{A}\Leftrightarrow \Im A=V$
    
    Докажем следующее утверждение:
    \begin{statement}
        $\mathcal{A} $ инъективно $ \Leftrightarrow \Ker A=\{0\}$
    \end{statement}
    \begin{proof}
        \begin{enumerate}
            \item[]
            \item[\circ] $\Rightarrow$: $\mathcal{A}(x)=0\And 
\mathcal{A}(0)=0; \mathcal{A}$ инъективно $\Rightarrow x=0$
            \item[\circ] $\Leftarrow$: Пусть $\Ker \mathcal{A}=0$ и 
$\mathcal{A}(x)=\mathcal{A}(y)\Leftrightarrow 
\mathcal{A}(x)-\mathcal{A}(y)=0\Leftrightarrow 
\mathcal{A}(x-y)=0\Rightarrow x-y\in \Ker \mathcal{A} \Rightarrow x-y =0$, 
т.е. $x=y$
        \end{enumerate}
    \end{proof}
    Теперь вернемся к доказательству теоремы: инъективность 
$\Leftrightarrow \dim \Ker \mathcal{A}=0\Leftrightarrow \dim \Im 
\mathcal{A}=\dim V \Leftrightarrow \IM \mathcal{A} = V \Leftrightarrow $ 
сюръективность.
\end{proof}

\subsection{Вид общего решения}

Рассмотрим СЛУ $AX=B$ или $\mathcal{A}(x)=b$. Пусть знаем частное решение 
$X_0$ $($т.е. $AX_0=B)$. Тогда: 

$AX=B \Leftrightarrow AX=AX_0 \Leftrightarrow A(X-X_0)=0 \Leftrightarrow 
X-X_0\in \Ker \mathcal{A}$

Значит, знаем общий вид решения: $X=X_0+Y, Y\in \Ker \mathcal{A}$ $($или 
$\mathcal{X}=\mathcal{X}_0+\Ker \mathcal{A})$

\subsection{Транспонирование}

\begin{definition}
    Пусть $A\in M_{m,n}(K)=(a_{i,j})_{^{i=1..m}_{j=1..n}}$; тогда 
\textbf{транспонирование} $A\ -\ A^{T}\in M_{n,m}(K);\ 
A^T=(a'_{i,j})_{^{i=1..m}_{j=1..n}},\ $где $ (a'_{i,j})=a_{j,i}$ 
(отражение относительно главной диагонали).  
\end{definition}

\begin{statement}
    \textbf{Свойства:}

    \begin{enumerate}
        \item[\circ] $(A+B)^T=A^T+B^T;\ (kA)^T=k\cdot A^T\ -$ это линейное 
отображение.
        \item[\circ] $(A\cdot B)^T=B^T\cdot A^T$

        Если $\exists A^{-1}$, то $\exists (A^T)^{-1}=(A^{-1})^T$.
    \end{enumerate}
\end{statement}

\begin{proof}
    \begin{enumerate}
        \item[]
        \item Пусть $A\cdot B=C=(c_{i,j}),\ (A\cdot B)^T=(c'_{i,j})$

        $c'_{i,j}=c_{j,i}=\sum\limits_{s}a_{j,s}\cdot 
b_{s,i}=\sum\limits_{s}a'_{s,j}\cdot b_{i,s}=\sum\limits_{s}b'_{i,s}\cdot 
a'_{s,j}=(B^T\cdot A^T)_{i,j}$

        \item $\begin{cases} A\cdot B = E \\ B\cdot A = E
        \end{cases},\ B=A^{-1}\Rightarrow B^T\cdot A^T=(A\cdot B)^T=E^T=E$
    \end{enumerate}
\end{proof}

\begin{statement}
    \textbf{Свойства ранга:}

    \begin{enumerate}
        \item $\rk A= \rk A^T$ (если строки и столбцы поменять ролями, то 
поменяем ролями, то $\rk$ не изменится)
        \item $\rk A\cdot B\leq \min(\rk A, \rk B)$
        \item $\rk (A+B)\leq \rk A+ \rk B$
        \item $A\in M_n(K),\ \rk A=n\Leftrightarrow A\ -$ обратима.
    \end{enumerate}
\end{statement}

\begin{proof}
    \begin{enumerate}
        \item[]
        \item[2.] $\rk A=\dim \Im (X\rightarrow AX)$

        $U\xrightarrow{\mathcal{B}}V\xrightarrow{\mathcal{A}}W$

        \begin{enumerate}
            \item[a)] $\rk AB=\dim \Im \{(AB)X\mid X\in U \}\leq \dim 
\{AY\mid Y\in V\}=\rk A\ (\{ABX\}\subset\{AY\})$
            \item[b)] $\rk AB=\dim \Im (A\circ B)=\dim (A\circ B(U))=\dim 
(A(\underbrace{B(U)}_{=\Im B}))=\dim (\Im A\mid_{\Im B})\leq \dim \Im B$
        \end{enumerate}
        \item[1.] Знаем, что $\exists C,D:\ 
CAD=\left(\begin{array}{c|c}E_r & 0\\\hline 0 & 0\end{array}\right)$

        $D^T\cdot A^T\cdot C^T=(CAD)^T=\left(\begin{array}{c|c}E_r & 
0\\\hline 0 & 0\end{array}\right)^T=\left(\begin{array}{c|c}E_r & 
0\\\hline 0 & 0\end{array}\right)$ (размеры строки и столбца из нулей 
поменялись местами)

        $\rk A=r,\ \rk A^T=r,$ т.к.:
        \begin{enumerate}
            \item[1)] $\rk (D^T\cdot A^T\cdot C^T)\leq \rk A^T$ по 
свойству 2.
            \item[2)] $\rk A^T=\rk ((D^T)^{-1}(D^T\cdot A^T\cdot 
C^T)(C^T)^{-1})\leq \rk (D^T\cdot A^T\cdot C^T)$ по свойству 2.
        \end{enumerate}
        \item[3.] упр.
        \item[4.] $\Rightarrow:\ A$ обратима $\Rightarrow n=\rk 
E=\rk(A\cdot A^{-1})\leq \rk A\Rightarrow \rk A=n$ (т.к. $\rk A\leq n$)

        $\Leftarrow:\ \rk A=n,\ \exists C,D:\ 
CAD=\left(\begin{array}{c|c}E_r & 0\\\hline 0 & 0\end{array}\right),$ где 
$r=\rk A\Rightarrow CAD=E\Rightarrow CA=D^{-1}\Rightarrow DCA=E$

        Аналогично $ADC=E\Rightarrow DC=A^{-1}$.
    \end{enumerate}
\end{proof}

\begin{corollary}
    Ранг по строкам равен рангу по столбцам (по первому свойству).
\end{corollary}

\begin{statement}
    $A\in M_n(K)$; следующие условия равносильны:
    \begin{enumerate}
        \item $A\ -$ обратима.
        \item $\Ker A=\{0\}$
        \item $\Im A=K^n$
        \item Строки $A$ линейно независимы.
        \item Столбцы $A$ линейного независимы.
    \end{enumerate}
\end{statement}

\begin{designation}
    $(M_n(K))^*=GL(n,K)\ -$ полная линейная группа.
\end{designation}

\begin{statement}
    Все есть матрица.
\end{statement}

\begin{example}
\textbf{Матричная реализация: $\R,\ \C$}

\begin{statement}
    Подмножество $M_2(\R),$ состоящее из матриц вида $\begin{pmatrix}
        a & b \\ b & a
    \end{pmatrix}$ является подкольцом и полем.
\end{statement}

Оно изоморфно $\C:\ P(a+bi)=\begin{pmatrix}
    a & b \\ -b & a
\end{pmatrix}$

Объяснение: $\C\ -$ векторное пространство над $\R$ размерности 2 $\langle 
1,i\rangle $

$a+bi\leadsto \begin{pmatrix}
    a \\ b
\end{pmatrix},\ x_1+x_2i\mapsto \begin{pmatrix}
    x_1 \\ x_2
\end{pmatrix}$

Умножение на $z\in \C\ -$ линейное отображение: $(x_1+x_2)i=-x_2+x_1i$

$\begin{pmatrix}
    x_1 \\ x_2
\end{pmatrix}\leadsto \begin{pmatrix}
    -x_2 \\ x_1
\end{pmatrix}=\begin{pmatrix}
    0 & -1 \\ 1 & 0
\end{pmatrix}\begin{pmatrix}
    x_1 \\ x_2
\end{pmatrix}$

\end{example}
