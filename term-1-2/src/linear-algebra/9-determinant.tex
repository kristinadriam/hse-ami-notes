\section{Определитель}
\subsection{Определитель}

\begin{definition}
    \textbf{Определитель} $-$ функция det: $M_n(K)\rightarrow K$.
\end{definition}

\begin{example}
    Площадь/объем.

    (картиночка)

    Оказывается, что $S=|ad-bc|$. Как к этому прийти?

    Свойства площади:
    \begin{enumerate}
        \item[1)] (еще картиночка)

        $S(\Vec{u}, k\Vec{v})=kS(\Vec{u}, \Vec{v})$

        \item[2)] (еще картиночка)

        $S(\Vec{u}, \Vec{v}+\Vec{w})=S(\Vec{u}, \Vec{v})+S(\Vec{u}, \Vec{w})$

        \item[2')] $S(\Vec{u}, \Vec{v}+ k\Vec{u})=S(\Vec{u}, \Vec{v})=S(\Vec{u}, \Vec{v})+kS(\Vec{u}, \Vec{u})$ (1 и 2 свойства)

        \item[3)] $S(\Vec{u}, \Vec{u})=0$

    \end{enumerate}

    Теперь можно: (опяяяять картинки)

    Воспользуемся еще одной аксиомой, что $S($квадрат 1 на 1$)=1:(ad-bc)S\begin{pmatrix}
                                                                             1 & 0 \ 0 & 1
    \end{pmatrix}=ad-bc$.
\end{example}

\subsection{Полилинейная функция}
\begin{definition}
    $K^n$ (или $V$) $-$ векторное пространство над $K$.

    Функция $f:\underbrace{V\times V\times ... \times V}_{m}\rightarrow K$ называется \textit{полилинейной}, если:

    $\forall i\ v_{1},...,v_{i-1}, v_{i+1},...,v_m\in K^n\ f(v_{1},...,v_{i-1}, v_{i+1},...,v_m):V\rightarrow K$ линейная, то есть:

    $f(v_{1},...,v_{i-1}, v_i+b_iv'_i,...,v_m)=f(v_{1},...,v_{i},...,v_m)+bf(v_{1},...,v'_{i},...,v_m)$.
\end{definition}

\begin{example}
    $m=1:$ полилинейное = линейное.

    $m=2:$ пример: скалярное произведение векторов.
\end{example}

\begin{definition}
    Полилинейная функция $f$ называется кососимметричной, если верно, что $v_i=v_j\Rightarrow f(v_1,..., v_m)=0$.
\end{definition}

\begin{remark} (покажем на случае функции от двух параметров, общий случай выводится аналогично)

    $f(x, y)\ -$ полилинейная.
    \begin{enumerate}
        \item $f$ кососимметричная.
        \item $f(x,y)=-f(y,x)\ \forall x, y$.
    \end{enumerate}
    Тогда $1\Rightarrow 2$ и $1\Leftrightarrow 2$, если $\text{char} K\neq 2$.
\end{remark}

\begin{proof}
    $2\Rightarrow 1:\ f(x,x)=-f(x,x)\Rightarrow 2f(x,x)=0\overset{\text{если char} K\neq 2}{\Rightarrow} f(x,x)=0$.

    $1\Rightarrow 2: 0=f(x+y, x+y)=\underbrace{f(x,x)}_{=0}+f(x,y)+f(y,x)+\underbrace{f(y,y)}_{=0}\Rightarrow f(x,y)=-f(y,x)$.
\end{proof}

\begin{statement}
    Кососимметричная полилинейна функция однозначно задается значениями $f(e_{i_1},...,e_{i_m})$, где $i_1<...<i_n$.
\end{statement}

\begin{proof}
    Достаточно вычислить $f_1(e_{i_1},...,e_{i_m})$ (по предыдущему утверждению).

    $\exists k,l:\ i_l=i_k\Rightarrow f(...)=0$ (по определнию)

    Пусть $i_1\neq i_2\neq ... \neq i_m;\ i_k\ -\ \min;\ f(e_{i_1},...,e_{i_k}...,e_{i_m})=-f(e_{i_k}, e_{i_1},...,e_{i_m})$. Продолжая это, получим $f(e_{i_1},...,e_{i_m})=\pm f(e_{j_1},...,e_{j_m})$, где $j_1<...<j_m$.
\end{proof}

\begin{definition}
    Пусть $n=m$; тогда билинейная кососимметричная функция $f$ называется \textit{определителем порядка} $n$ ($f\neq 0$).
\end{definition}

\begin{theorem}
    $f_1$ и $f_2\ -$ определители $\Rightarrow \exists c\in K^*:f_1=c\cdot f_2$.
\end{theorem}

\begin{proof}
    Пусть $f_1(e_1,...,e_n)=c\cdot f_2(e_1,...,e_n)$.

    Тогда: $\forall e_{i_1},...,e_{i_n}\ f_1(e_{i_1},...,e_{i_n})= c\cdot f_2(e_{i_1},...,e_{i_n})$.

    Тогда по утверждению 1 $f_1 \cdot сf_2$ всегда.
\end{proof}

\begin{definition}
    $V=K^n$; определителем будем называть такой определитель, что:

    $f\bigg(\begin{pmatrix}
                1 \\ 0 \\ .. 0
    \end{pmatrix}, \begin{pmatrix}
                       0 \\ 1 \\ .. 0
    \end{pmatrix}, ..., \begin{pmatrix}
                            0 \\ 0 \\ .. 1
    \end{pmatrix} \bigg)=1$. Она обозначается $\det$.
\end{definition}


\subsection{Четность перестановки}
\begin{definition}
    Перестановка $s\in S_n,\ s_i:\{1,...,n\}\rightarrow \{1,...,n\}$ биекция.
\end{definition}

\begin{statement}
    $s\ -$ композиция трансвеций.
\end{statement}

\begin{proof}
    Индукция по $n$.

    Переход: $n\rightarrow n + 1:\pi\in S_{n+1},\ ,\pi(n+1)=x$

    Рассмотрим $s_{n+1, x}\circ \pi=\overline{\pi};\ \overline{\pi}(n+1)=,\pi(x)=n+1\overset{\text{забываем про $n$}}{\Rightarrow}\overline{\pi}\mid_{\{1,...,n\}}$ по и.п. это $s_{i_1,j_1}\circ...\circ s_{i_k,j_k}\Rightarrow \pi = s_{n+1,x}\circ s_{i_1,j_1}\circ...\circ s_{i_k, j_k}$
\end{proof}

\begin{definition}
    \textbf{Четность перестановки}

    Если $\pi=s_1\circ ...\circ s_{2k},$ где $ s_i\ -$ транспозиции, то $\pi\ -$ \textit{четная} перестановка.

    Если $\pi=s_1\circ ...\circ s_{2k+1},$ где $ s_i\ -$ транспозиции, то $\pi\ -$ \textit{нечетная} перестановка.
\end{definition}

\begin{theorem}
    \textbf{Явная формула для определителя}
    $A=(a_{i,j})_{i,j=1..n}$. Тогда функция $\det A=\sum\limits_{\pi\in S_n}(-1)^{\varepsilon(\pi)}a_{1,,\pi(1)}\cdot a_{2,,\pi(2)}\cdot...\cdot a_{n,,\pi(n)}$, где $\varepsilon(\pi)\ -$ четность перестановки $\pi$ полилинейная, кососимметричная и нормальная (т.е. $\det$ в прежнем смысле).
\end{theorem}

\begin{remark}
    Эта формула $-$ сумма ладейных произведеинй, взятых со знаком.
\end{remark}

\begin{example}
    $n=2:+: \begin{vmatrix} * &  \\ & *
    \end{vmatrix}, -: \begin{vmatrix} & *   \\ * &
    \end{vmatrix}\Rightarrow \det \begin{pmatrix}
                                      a & b  \\ c & d
    \end{pmatrix}=ad-bc$ (в общем случае будет $n!$ слагаемых)
\end{example}

\begin{definition}
    Четность через инверсии: $\Tilde{\varepsilon}(\pi)=\bigg|\{(i,j)\mid i<j \And ,\pi(i)<,\pi(j)\}\bigg|$.
\end{definition}

\begin{example}
    $\pi:\begin{vmatrix} 1 & 2 & 3 & 4 & 5 \\ 5 & 1 & 2 & 4 & 3
    \end{vmatrix}\rightarrow 5$ инверсий $\Rightarrow 5$ нечетно.
\end{example}

\begin{theorem}
    Два определения четности совпадают.
\end{theorem}

\begin{proof}
    Это следует из леммы.

    \begin{lemma}
        Если $t\ -$ транспозиция, то  у $\Tilde{\varepsilon}(t\pi)$ и $\Tilde{\varepsilon}(\pi)$ число инверсий по модулю 2 различно.
    \end{lemma}

    \begin{proof}
        По индукции: $\Tilde{\varepsilon}(s_1,...,s_{2k})= 0, \Tilde{\varepsilon}(s_1,...,s_{2k+1})= 0$

        База: $\Tilde{\varepsilon}(id)=0$

        Переход:
        $\pi:,\pi(1)...,\pi(k)=i...,\pi(l)=j...,\pi(n)$

        $\Tilde{\pi}=t_{i,j}\circ\pi:\Tilde{\pi}(1)...\Tilde{\pi}(k)=j...\Tilde{\pi}(l)=i...\Tilde{\pi}(n)$

        Какие пары поменяли статус: их $l-k+1$ с участием $k$, $l-k+1$ с участием $l$ и пара $(k,l)$. Итого: $2(l-k+1)+1ё -$ нечеттное число смен статуса $\Rightarrow$ четность поменялась.
    \end{proof}
\end{proof}

\begin{theorem}
    $\det\ -$ полилинейная кососиметричная функция, $\det(E)=1$ (то есть $\det$ в аксиоматическом смысле существует).
\end{theorem}

\begin{proof}
    \begin{enumerate}
        \item[]
        \item $\underline{\det(E)}=\det\begin{pmatrix}
                                           1 & & 0 \\ & \ddots & \\ 0 & & 1
        \end{pmatrix}=a_{11}a_{22}...a_{nn}\cdot(-1)^{\varepsilon(id)}=1\cdot...\cdot 1=1$

        Одна ненулевая ладейная расстановка.
        \item \underline{Полилинейность:} $A=(C_1\mid C_2\mid ...\mid C'_i + a\cdot C''_i \mid ...\mid C_n),\ A=(a_{ij}),\ a_{ki}=a'_{ki}+a\cdot a''_{ki}$

        Тогда $\forall \pi\in S_n:\ (-1)^{\varepsilon(\pi)}a_{1,\pi(1)}\cdot a_{2,\pi(2)}\cdot ... \cdot (a'_{\pi^{-1}(i)i}+a\cdot a''_{\pi^{-1}(i)i})\cdot ...\cdot a_{n,\pi(n)}=(-1)^{\varepsilon(\pi)}a_{1,\pi(1)}\cdot a_{2,\pi(2)}\cdot ... \cdot a'_{\pi^{-1}(i)i}\cdot ...\cdot a_{n,\pi(n)}+a\cdot (-1)^{\varepsilon(\pi)}a_{1,\pi(1)}\cdot a_{2,\pi(2)}\cdot ... \cdot a''_{\pi^{-1}(i)i}\cdot ...\cdot a_{n,\pi(n)}$

        Теперь сложим равенства для всех $\pi$ и получим линейность (по $i$-ому аргументу)

        \item \underline{Кососимметричность:} (!) $A=(C_1\mid C_2\mid ...\mid \overset{i}{C_i} \mid ...\mid \overset{j}{C_j} \mid ...\mid C_n)\Rightarrow \det A=0,\ a_{ki}=a_{kj}\ \forall k$

        Разобьем слагаемые в $\det A$ на пары $(\pi, t_{i,j}\circ \pi)$

        $(-1)^{\varepsilon(\pi)}\cdot a_{1,\pi(1)}\cdot ...\cdot a_{\pi^{-1}(i), i}\cdot ...\cdot a_{\pi^{-1}(j), j}\cdot ...\cdot a_{n,\pi(n)} + (-1)^{\varepsilon(t_{i,j}\circ\pi)}\cdot a_{1,\pi(1)}\cdot ...\cdot \overset{=a_{\pi^{-1}(i), i}}{a_{\pi^{-1}(i), j}}\cdot ...\cdot \overset{=a_{\pi^{-1}(j), j}}{a_{\pi^{-1}(j), i}}\cdot ...\cdot a_{n,\pi(n)}=\prod\limits_{k=1}^na_{k,\pi(k)}\underbrace{((-1)^{\varepsilon(\pi)}+(-1)^{\varepsilon(t_{i,j}\circ\pi)})}_{=0\text{ по лемме}}\Rightarrow$ весь $\det=0$.
    \end{enumerate}
\end{proof}

\begin{theorem}
    \textbf{Определитель и элементарные преобразования}
    \begin{enumerate}
        \item $\det(A\circ t_{i,j}(a))=\det A$.
        \item $\det(A\circ m_{i}(a))=a\cdot \det A$.
        \item $\det(A\circ s_{i,j})=-\det A$.
    \end{enumerate}
\end{theorem}

\begin{proof}
    \begin{enumerate}
        \item[3.] Это вторая формулировка кососимметричности: $f(x,y)=-f(y,x)$.
        \item[2.] Это линейность по $i$-ому столбцу.
        \item[1.] $A=(C_1\mid C_2\mid ...\mid C_n)$

        $\det (A\circ t_{j,i}(a))=\det(C_1\mid ...\mid C_i\mid ...\mid C_j+a\cdot C_i\mid ...\mid C_n)=\overset{\text{лин-ть по }j}{=}\underbrace{\det(C_1\mid ...\mid C_i\mid ...\mid C_j\mid ...\mid C_n)}_{=a\cdot \det A}+\underbrace{\det(C_1\mid ...\mid C_i\mid ...\mid C_i\mid ...\mid C_n)}_{=0}$
    \end{enumerate}
\end{proof}

\begin{corollary}
    $\det$ можно посчитать так: $A\overset{t_{i,j}}{\rightarrow}\Tilde{A}=\begin{pmatrix}
                                                                              \Tilde{a}_{11} & & * \\ & \ddots & \\ 0 & & \Tilde{a}_{nn}
    \end{pmatrix}$
\end{corollary}

\begin{proof}
    По th: $\det A=\det \Tilde{A};\ \det \Tilde{A}=\Tilde{a_{11}}\cdot ...\cdot \Tilde{a_{nn}}\ (\Tilde{a_{11}},...,\Tilde{a_{nn}}\ -$ единственная возможножно ненулевая ладейная расстановка)
\end{proof}

\begin{corollary}
    $A$ обратима $\Leftrightarrow \det A\neq 0$.
\end{corollary}

\begin{proof}
    $A$ обратима $\Leftrightarrow \Tilde{A}$ обратима $\Leftrightarrow$ все $\Tilde{a_{ii}}$ не равны $0\Leftrightarrow \det \Tilde{A}=\Tilde{a_{11}}\cdot ...\cdot \Tilde{a_{nn}}\neq 0$.
\end{proof}

\begin{theorem}
    $\det A=\det A^T$

    То есть любые свойства $\det$, верные для столбцов, верны и для строк и наоборот.
\end{theorem}

\begin{proof}
    $A=(a_{ij}),\ A^T=(a'_{ij}),\ a'_{ij}=a_{ji}$

    $(-1)^{\varepsilon(\pi)}a_{1,\pi(1)}\cdot ...\cdot a_{n,\pi(n)}=(-1)^{\varepsilon(\pi)}a_{\pi^{-1}(1),1}\cdot ...\cdot a_{\pi^{-1}(n),n}=(-1)^{\varepsilon(\pi)}a'_{1,\pi(1)}\cdot ...\cdot a'_{n,\pi(n)}\overset{(*)}{=}(-1)^{\varepsilon(\pi^{-1})}a_{1,\pi(1)}\cdot ...\cdot a_{n,\pi(n)}$

    Сложим по всем $\pi$ и получим $\det A=\det A^T$.

    $(*):\varepsilon(\pi)=\varepsilon(\pi^{-1})$, т.к. $\pi=t_1...t_k;\ \pi^{-1}=(t_1...t_k)^{-1}=t^{-1}...t^{-1}_1=t_k...t_1$ тоже $k$.
\end{proof}

\begin{theorem}
    \textbf{Разложение по строке/столбцу}

    $A=(a_{ij}),\ A_{i,j}\ -\ \det$ матрицы6 полученной из $A$ удалением $i$-ой строки и $j$-ого столюца (минор).

    Тогда $\forall i=1...n:$

    \begin{enumerate}
        \item $\det A=\sum\limits_{j=1}^n(-1)^{i+j}\cdot a_{i,j}\cdot A_{i,j}\ -$ по строке.
        \item $ \forall i=1...n:\ \det A=\sum\limits_{j=1}^n(-1)^{i+j}\cdot a_{j,i}\cdot A_{j,i}\ -$ по столбцу.
    \end{enumerate}

    Выразим $\det A$ через $n$ определителей $(n-1)$-ого порядка.
\end{theorem}

\begin{proof}
(для строки)

    $i$-ая строка: $(a_{i,1},...,a_{i,n})=\sum\limits_{j=1}^n a_{i,j}(0,...,0,\overset{j}{1},0,...,0)$

    Тогда по линейности $\det A=\sum a_{i,j}\cdot \det \begin{pmatrix}
                                                           a_{1,1} & \overset{j}{...} & a_{1,n} \\ 0 & ..\ 1\ .. & 0 \\
    \end{pmatrix}\ i=($переставим $a_{i,j}$ на позицию $(1,1)$ $i-1$ транспозиции строк и $j-1$ транспозиции столюцов$)=\sum a_{i,j}\cdot  (-1)^{i+j}\cdot \det \left(
    \begin{array}{c|ccc}
        1 & 0 & ... & 0\\
        \hline
        & & \Tilde{A}_{i,j}
    \end{array}
    \right)=\sum a_{i,j}\cdot  (-1)^{i+j}\cdot \det (\Tilde{A}_{i,j})=\sum a_{i,j}\cdot  (-1)^{i+j}\cdot A_{i,j}$
\end{proof}

\begin{example}
    $\det\begin{pmatrix}
             a & b & c \\ d & e & f \\ g & h & i
    \end{pmatrix}=a\cdot \det \begin{pmatrix}
                                  e & f \\ h & i
    \end{pmatrix}-b\cdot \det \begin{pmatrix}
                                  d & f \\ g & i
    \end{pmatrix}+c\cdot \det \begin{pmatrix}
                                  d & e \\ g & h
    \end{pmatrix}$
\end{example}

\begin{corollary}
    Пусть $i\neq i'$. Тогда $\sum\limits_{j=1}^n (-1)^{i+j}a'_{i,j}\cdot A_{i,j}=0\ (*)$.
\end{corollary}

\begin{proof}
    По предыдущей th $*\ -$ определитель $A'\ -$ матрица, у которой $i$-ая строка заменена на $i'$-ую строку; $\det A'=0$, т.к. есть совпадающие строки.
\end{proof}

\begin{definition}
    $A^{adj}=((-1)^{i+j}A_{j,i})_{i=1...n,\ j=1...n}\ -$ \textit{присоединенная матрица} к $A$.
\end{definition}

\begin{statement}
    $A\cdot A^{adj}=A^{adj}\cdot A=(\det A)\cdot E$
\end{statement}

\begin{proof}
    Это предыдущая теорема и ее следствия:

    th: у $A\cdot A^{adj}$ на диагонали стоит $\det A$.

    cons: у $A\cdot A^{adj}$ вне диагонали стоят 0.

    $\sum\limits_j (-1)^{i+j}\cdot a'_{i,j}\cdot A_{i,j}=\sum\limits_j (A_{i,j})(A^{adj})_{j,i}$
\end{proof}

\begin{corollary}
    $\det A\neq 0$. Тогда $A^{-1}=\frac{1}{\det A}\cdot A^{adj}$.
\end{corollary}

\begin{theorem}
    \textbf{Формула Крамера}

    $AX=B\ -$ СЛУ, $A\in M_n(K),\ \det A\neq 0\ (\Leftrightarrow $решение существует и единственно $\forall B)$. Тогда $X=\begin{pmatrix}
                                                                                                                              x_1 \\ ... x_n
    \end{pmatrix},$ где $x_i=\frac{\Delta_i}{\Delta}$, где $\Delta=\det A,\ \Delta_i=\det$ матрицы, полученной из $A$ заменой $i$-ого столбца на столбец $B$.
\end{theorem}

\begin{proof}
    $\begin{pmatrix}
         x_1 \\ ... x_n
    \end{pmatrix} = A^{-1}\cdot \begin{pmatrix}
                                    b_1 \\ ... \\ b_n
    \end{pmatrix}$

    $x_i=\sum\limits_{k=1}^n(A^{-1})_{i,k}\cdot b_k=\frac{1}{\det A}\cdot \sum\limits_{k=1}^nA^{adj}_{i,k}\cdot b_k=\frac{1}{\Delta}\cdot \sum\limits_{k=1}^n(-1)^{i+k}\cdot A_{k,i}\cdot b_k\overset{\text{разложение по $i$-ому столбцу}}{=}\begin{pmatrix}
                                                                                                                                                                                                                                                  a_{1,1} & ... & \overset{i}{b_1} & ... & a_{1,n} \\
                                                                                                                                                                                                                                                  \\
                                                                                                                                                                                                                                                  a_{n,1} & ... & b_k & ... & a_{n,n}
    \end{pmatrix}$

    $x_i=\frac{\Delta_i}{\Delta}$
\end{proof}

\begin{definition}
    $A_{k,i}$ называется \textit{алгебраическим дополнением}. $a_{k,i}\ -$ \textit{миноры} $(n-1)$-ого порядка.
\end{definition}

\begin{theorem}
    \textbf{Явная формула для ранга: минорный ранг}

    Ранг матрицы равен порядку максимального ненулевого минора, т.е. размеру max подматрицы с ненулевым определителем.
\end{theorem}

\begin{proof}
    \begin{enumerate}
        \item[]
        \item Столбцовый ранг $\geq$ минорного.

        $\exists$ ненулевой минор $k\times k$. НУО это первые $k$ строк и $k$ столбцов: $\left(\begin{array}{c|c}
                                                                                                   \Tilde{A} & \\
                                                                                                   \hline
                                                                                                   &
        \end{array}\right)$

        $\det \Tilde{A}\neq 0\Rightarrow \Tilde{A}$ обратима $\Rightarrow$ столбцы $\Tilde{A}\ -$ ЛНЗ $\Rightarrow k$ столбцов $A$ ЛНЗ $\Rightarrow\rk A\geq k$.

        \item Столбцовый ранг $\leq$ минорного.

        Пусть $\rk A=k$. НУО первые $k$ столбцов ЛНЗ: $\left(\begin{array}{c|c}
                                                                 \Tilde{A} &
        \end{array}\right)$

        $\rk \Tilde{A}=k\Rightarrow$ в $\Tilde{A}$ есть $k$ ЛНЗ строк, т.е. подматрица $\Tilde{\Tilde{A}}\ k\times k$ с ЛНЗ строками $\Rightarrow \Tilde{\Tilde{A}}$ обратима $\Rightarrow\det\Tilde{\Tilde{A}}\neq 0$

        Нашли ненулевой минор $k$-ого порядка $\Rightarrow$ минорный ранг $\geq \rk A$.
    \end{enumerate}
\end{proof}